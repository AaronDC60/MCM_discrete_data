\section{Theoretical background}

This section covers the theoretical concepts necessary to understand the research project.

\subsection{Spin models}

One approach to represent a system is through a spin model. Assume a given system with $n$ components where a single observation of the system yields one value for every component.
Such an observation can be seen as a specific state of the system.

\subsubsection{Binary spin operators}

In case that the observations can be binarized, the state of the system can be written as a spin configuration,

\begin{equation}
    \mathbf{s} = (s_1 \dots s_n),
\end{equation}

\noindent
of $n$ spin variables, where every variable takes a value of either +1 or -1.

\noindent
In a spin configuration, we can express an interaction between a subset $\mu$ of these spins using a spin operator which is defined as the product of all spins in the subset,

\begin{equation}\label{eq:spin_op}
    \phi^\mu(\mathbf{s}) = \prod_{i \in \mu} s_i.
\end{equation}

\noindent
Given that all spin values are either +1 or -1, the spin operator will be +1 if an even number of spins are -1 and it will be -1 if an odd number of spins are -1.
The sum over all possible spin configurations of a spin operator is equal to zero,

\begin{equation}\label{eq:sum_over_conf}
    \sum_{\mathbf{s} \in \mathbf{S}} \phi^\mu(\mathbf{s}) = 0,
\end{equation}

\noindent
because half of the spin configurations can be written as a spin configuration in the other half with exactly one spinflip.
In a system with $n$ spin variables, there are $2^n - 1$ possible subsets, which means that there are $2^n - 1$ different spin operators.

\begin{definition}
    A set of operators, $\Omega$ is called orthogonal if

    \begin{equation}\label{eq:ortho}
        \frac{1}{2^n} \sum_{\mathbf{s} \in \mathbf{S}} \phi^\mu(\mathbf{s}) \phi^\nu(\mathbf{s}) = \delta_{\mu\nu} \qquad \forall \;\; \phi^\mu, \phi^\nu \in \Omega
    \end{equation}
\end{definition}

\begin{definition}
    A set of operators, $\Omega$ is called complete if

    \begin{equation}\label{eq:complete}
        \frac{1}{2^n} \sum_{\mu \in \Omega} \phi^\mu(\mathbf{s}) \phi^\mu(\mathbf{s}^\prime) = \delta_{\mathbf{s}\mathbf{s}^\prime},
    \end{equation}
    as this yields a unique relation between the spin configuration and the values for every operator in the set.
\end{definition}

\noindent
If we define the identity operator, $\phi^0 = 1$, then the set of operators $\Omega_n = \{\phi^\mu(\mathbf{s})\}_{\mu \in \{0 \dots 2^n-1\}}$ is a complete and orthogonal set.
For orthogonality, we can rewrite Equation \ref{eq:ortho}, use Equation \ref{eq:sum_over_conf} if $\mu$ and $\nu$ are different and use the definition of the identity operator if $\mu$ and $\nu$ are the same.

\begin{align*}
    \frac{1}{2^n} \sum_{\mathbf{s} \in \mathbf{S}} \phi^\mu(\mathbf{s}) \phi^\nu(\mathbf{s}) &= \frac{1}{2^n} \sum_{\mathbf{s} \in \mathbf{S}} \phi^{\mu \oplus \nu (\mathbf{s})}\\
    &= \delta_{\mu\nu}
\end{align*}

\noindent
To show the completeness of $\Omega_n$, we can plug Equation \ref{eq:spin_op} into Equation \ref{eq:complete} and rewrite the sum over all spin operators as a product of sums using the bitstring representation of the spin operators.

\begin{align*}
    \frac{1}{2^n} \sum_{\mu \in \Omega_n} \phi^\mu(\mathbf{s}) \phi^\mu(\mathbf{s}^\prime) &= \frac{1}{2^n} \sum_{\mu=0}^{2^n-1} \prod_{i \in \mu} s_i s_i^\prime \\
    &= \frac{1}{2^n} \sum_{\alpha_1 = 0,1} \dots \sum_{\alpha_n = 0,1} \prod_{i = 1}^{n} (s_i s_i^\prime)^{\alpha_i} \\
    &= \delta_{\mathbf{s}\mathbf{s}^\prime}
\end{align*}

\noindent
If the two spin configurations are the same, $s_i s_i^\prime$ will always be equal to 1 while it will be half of the time +1 and half of the time -1 for two configurations that are different.

\begin{definition}
    A set of $n$ spin operators that can fully generate $\Omega_n$ by taking linear combinations of the individual spin operators is called a generating set of $\Omega_n$.
\end{definition}

\begin{definition}
    A set of spin operators that for which the product of every subset does not yield the identity operator is called a set of independent operators.
\end{definition}

\subsubsection{Discrete spin operators}

In case a system of $n$ variables where each variable can have $q$ different values, labeled from $0$ to $q-1$, a state of the system is a vector $\boldsymbol{\alpha}= (\alpha_1 \dots \alpha_n) \in {(\mathbb{Z}/q\mathbb{Z})}^n$.
We can still see these states as spins by representing them as vectors on a unit circle with an angle of $\frac{2\pi}{q}\alpha$.
The mapping between the value $\alpha_j$ of a variable and the spin value $s_j$ is then given by the following relation

\begin{equation}
    s_j = e^{\frac{2\pi i \alpha_j}{q}}.
\end{equation}

\begin{figure}[h]
    \centering
    \begin{tikzpicture}[scale=4,cap=round,>=latex]
        % draw the coordinates
        \draw[dashed] (-1.5cm,0cm) -- (1.5cm,0cm) node[right,fill=white] {};
        \draw[dashed] (0cm,-1.5cm) -- (0cm,1.5cm) node[above,fill=white] {};

        % draw the unit circle
        \draw[thick] (0cm,0cm) circle(1cm);

        \foreach \x in {0,1,2} {
            % lines from center to point
            \draw[gray] (0cm,0cm) -- (\x*120:1cm);
            % dots at each point
            \filldraw[black] (\x*120:1cm) circle(0.4pt);
            % draw each angle in degrees
            \draw (\x*120:0.6cm) node[fill=white] {$\alpha = \x$};
    }
    \foreach \x/\xtext/\y in {
        % alpha = 1
        120/\text{cos}\left(\frac{2\pi}{3}\right)/\text{sin}\left(\frac{2\pi}{3}\right),
        % alpha = 2
        240/\text{cos}\left(\frac{4\pi}{3}\right)/\text{sin}\left(\frac{4\pi}{3}\right)}
            \draw (\x:1.25cm) node[fill=white] {$\left(\xtext, \y\right)$};

        % draw the horizontal and vertical coordinates
        \draw (-1.25cm,0cm) node[above=1pt] {$(-1,0)$}
            (1.25cm,0cm)  node[above=1pt] {$(1,0)$}
            (0cm,-1.25cm) node[fill=white] {$(0,-i)$}
            (0cm,1.25cm)  node[fill=white] {$(0,i)$};
    \end{tikzpicture}
    \caption{All states in a 3-state system drawn as the cube roots of unity.}
    \label{fig:roots_of_unity}
\end{figure}

\noindent
These spin values are equivalent to the q-th roots of unity, which are drawn in Figure \ref{fig:roots_of_unity} for q equal to three.
Analogous to the binary case, we can define a spin operator,

\begin{equation}
    \phi^{\boldsymbol{\mu}}(\mathbf{s}) = \prod_{j=1}^{n} s_j^{\mu_j},
\end{equation}

\noindent
with $\boldsymbol{\mu} \in {(\mathbb{Z}/q\mathbb{Z})}^n$. With the bijective relation between $\mathbf{s}$ and $\boldsymbol{\alpha}$ we can also write the spin operators in terms of $\boldsymbol{\alpha}$,

\begin{equation}
    \phi^{\boldsymbol{\mu}}(\boldsymbol{\alpha}) = e^{\frac{2\pi i}{q} \sum_{j=1}^{n}  \alpha_j \mu_j},
\end{equation}

\noindent
In total there are $q^n$ spin operators that form a finite multiplicative group that is equal to the set of the q-th roots of unity.
Applying a spin operator $\phi^{\boldsymbol{\mu}}$ on a state $\boldsymbol{\alpha}$ can be seen as a rotation around the unit circle starting from the zero state over an angle $\sum_{j=1}^{n}  \alpha_j \mu_j$.
Similar as in the binary case, we can construct a $q$ x $q$ spin operator matrix.

\begin{equation}
    \mathbf{S}^{(q)} = \begin{bmatrix}
        \phi^0(\alpha=0) & \phi^1(\alpha=0) & \hdots &  \phi^{q-1}(\alpha=0)\\
        \phi^0(\alpha=1) & \phi^1(\alpha=1) & \hdots &  \phi^{q-1}(\alpha=1)\\
        \vdots & \vdots & \ddots & \vdots \\
        \phi^0(\alpha=q-1) & \phi^1(\alpha=q-1) & \hdots & \phi^{q-1}(\alpha=q-1)\\
    \end{bmatrix}\label{eq:spin_op_matrix}
\end{equation}

\subsubsection{Probability distribution}

\noindent
It was previously mentioned that a spin model can be used to represent system. The idea is that we assume the system to be in some sort of an equilibrium and the observed spin configurations can be seen as sampled from a given probability distribution.
Usually a probability distribution with an exponential form is chosen,

\begin{equation}\label{eq:prob_distr}
    p(\boldsymbol{\alpha}| \mathbf{g}, \mathcal{M}) = \frac{1}{Z(\mathbf{g}, \mathcal{M})} e^{\sum_{\mu \in \mathcal{M}} g_\mu \phi^\mu(\boldsymbol{\alpha})},
\end{equation}

\noindent
with

\begin{equation}
    Z(\mathbf{g}, \mathcal{M}) = \sum_{\boldsymbol{\alpha} \in {(\mathbb{Z}/q\mathbb{Z})}^n}e^{\sum_{\mu \in \mathcal{M}} g_\mu \phi^\mu(\boldsymbol{\alpha})}
\end{equation}

\noindent
The model $\mathcal{M}$ is defined as a subset of $\Omega_n \setminus \{ \phi^0 \}$ and $g_\mu$ a parameter that indicates the strength of the interaction between the spin variables in $\mu$.
If a subset of spin variables is correlated, then we would expect specific patterns in the corresponding subset of the spin configuration to occur often in the data.
For example, many observations with values +1,-1 and -1,+1 for two spins that are negatively correlated. 
Then, the model to describes this system would have to contain the spin operator that corresponds to this subset with a large positive parameter to result in an increased probability of configurations with those patterns.

\noindent
The probability distribution in Equation \ref{eq:prob_distr} is similar to the Boltzmann distribution, which is used in statistical mechanics to describe a system at thermal equilibrium.
The derivation in Appendix \ref{sec:max_entropy} shows that this distribution can also be seen as the distribution that maximizes the entropy under the constraint that the expected value for every spin operator in the model is equal to their value in the observed data.

\noindent
If we set $g_\mu = 0 \quad \forall \mu \notin \mathcal{M}$ and $g_0 = -\log Z(\mathbf{g}, \mathcal{M})$, we can write the probability distribution as

\begin{equation}
    p(\boldsymbol{\alpha}) =e^{S(\boldsymbol{\alpha})}
\end{equation}

\noindent
with

\begin{align}
    S(\boldsymbol{\alpha}) &= \sum_{\mu \in {(\mathbb{Z}/q\mathbb{Z})}^n} g_\mu \phi^\mu(\boldsymbol{\alpha}),\notag \\
    &= \bigotimes_{i = 1}^{n} \mathbf{S}^{(q)} \cdot \mathbf{g}.\label{eq:s_alpha}
\end{align}

\noindent
In Appendix \ref{sec:2spin_2states} a small example is worked out for a 2-state,2-spin system.
Note that in the case with more than two states, $\phi^{\boldsymbol{\mu}}(\boldsymbol{\alpha})$ can be a complex number. However, $S(\boldsymbol{\alpha})$ has to be a real function to get real, positive probabilities.
In order to guarantee $S(\boldsymbol{\alpha})$ being real, the following constraint is set for the model parameters.

\begin{equation}
    g_{-\boldsymbol{\mu}} = g_{\boldsymbol{\mu}}^*.
\end{equation}

\noindent
We can also split the model parameter into a real and imaginary part,

\begin{equation}
    g_{\boldsymbol{\mu}} = a_{\boldsymbol{\mu}} + i b_{\boldsymbol{\mu}},
\end{equation}

\noindent
in which case the constraint becomes

\begin{equation}
    \begin{cases}
        a_{-\boldsymbol{\mu}} = a_{\boldsymbol{\mu}}\\
        b_{-\boldsymbol{\mu}} = -b_{\boldsymbol{\mu}}
    \end{cases}
    .
\end{equation}

\noindent
An alternative is to write the model parameters in polar coordinates,

\begin{equation}
    g_{\boldsymbol{\mu}} = r_{\boldsymbol{\mu}} e^{i \theta_{\boldsymbol{\mu}}}.
\end{equation}

\noindent
Then, the constraints are

\begin{equation}
    \begin{cases}
        r_{-\boldsymbol{\mu}} = r_{\boldsymbol{\mu}}\\
        \theta_{-\boldsymbol{\mu}} = -\theta_{\boldsymbol{\mu}}
    \end{cases}
    .
\end{equation}


\subsubsection{Example: 3-state,1-spin system}

In case of a 3-state system with 1 spin, there are three possible states and three spin operators.
The spin operator matrix, defined in Equation \ref{eq:spin_op_matrix}, for this system is 

\begin{equation}
    \mathbf{S}^{(3)} = \begin{bmatrix}
        \phi^0(\alpha=0) & \phi^1(\alpha=0) & \phi^2(\alpha=0)\\
        \phi^0(\alpha=1) & \phi^1(\alpha=1) & \phi^2(\alpha=1)\\
        \phi^0(\alpha=2) & \phi^1(\alpha=2) & \phi^2(\alpha=2)\\
    \end{bmatrix} = \begin{bmatrix}
        1 & 1 & 1\\
        1 & e^{\frac{2\pi i}{3}} & e^{\frac{4\pi i}{3}}\\
        1 & e^{\frac{4\pi i}{3}} & e^{\frac{2\pi i}{3}}
    \end{bmatrix}.
\end{equation}

\noindent
Using Equation \ref{eq:s_alpha}, we can express the function, $S(\alpha)$ for all three states,

\begin{align*}
    S(\alpha=0) &= g_1 \phi^1(\alpha=0) + g_2 \phi^2(\alpha=0),\\
    S(\alpha=1) &= g_1 \phi^1(\alpha=1) + g_2 \phi^2(\alpha=1),\\
    S(\alpha=2) &= g_1 \phi^1(\alpha=2) + g_2 \phi^2(\alpha=2).
\end{align*}

\noindent
Plugging in the values for the spin operators gives,

\begin{align*}
    S(\alpha=0) &= g_1 + g_2,\\
    &= a_1 + i b_1 + a_2 + i b_2,\\
    S(\alpha=1) &= g_1 e^{\frac{2\pi i}{3}} + g_2 e^{\frac{4\pi i}{3}},\\
    &= (a_1 + i b_1) \left[ \cos\left(\frac{2\pi}{3}\right) + i \sin\left(\frac{2\pi}{3}\right)\right] + (a_2 + i b_2) \left[ \cos\left(\frac{4\pi}{3}\right) + i \sin\left(\frac{4\pi}{3}\right)\right],\\
    S(\alpha=2) &= g_1 e^{\frac{4\pi i}{3}} + g_2 e^{\frac{2\pi i}{3}},\\
    &= (a_1 + i b_1) \left[ \cos\left(\frac{4\pi}{3}\right) + i \sin\left(\frac{4\pi}{3}\right)\right] + (a_2 + i b_2) \left[ \cos\left(\frac{2\pi}{3}\right) + i \sin\left(\frac{2\pi}{3}\right)\right].
\end{align*}

\noindent
Then, we can split the real and imaginary part in the exponent and use the constraints on the model parameters.

{\footnotesize
\begin{align*}
    S(\alpha=0) &= a_1 + i b_1 + a_1 - i b_1,\\
    &= 2 a_1,\\
    S(\alpha=1) &= (a_1 + i b_1) \left[ \cos\left(\frac{2\pi}{3}\right) + i \sin\left(\frac{2\pi}{3}\right)\right] + (a_1 - i b_1) \left[ \cos\left(\frac{4\pi}{3}\right) + i \sin\left(\frac{4\pi}{3}\right)\right],\\
    &= a_1 \left[\cos\left(\frac{2\pi}{3}\right) + i \sin\left(\frac{2\pi}{3}\right) + \cos\left(\frac{4\pi}{3}\right) + i \sin\left(\frac{4\pi}{3}\right)\right] + b_1 \left[ i \cos\left(\frac{2\pi}{3}\right) - \sin\left(\frac{2\pi}{3}\right) - i \cos\left(\frac{4\pi}{3}\right) + \sin\left(\frac{4\pi}{3}\right) \right], \\
    &= 2 a_1 \cos\left(\frac{2\pi}{3}\right) - 2b_1 \sin\left(\frac{2\pi}{3}\right), \\
    S(\alpha=2) &= (a_1 + i b_1) \left[ \cos\left(\frac{4\pi}{3}\right) + i \sin\left(\frac{4\pi}{3}\right)\right] + (a_1 - i b_1) \left[ \cos\left(\frac{2\pi}{3}\right) + i \sin\left(\frac{2\pi}{3}\right)\right],\\
    &= a_1 \left[\cos\left(\frac{4\pi}{3}\right) + i \sin\left(\frac{4\pi}{3}\right) + \cos\left(\frac{2\pi}{3}\right) + i \sin\left(\frac{2\pi}{3}\right)\right] + b_1 \left[ i \cos\left(\frac{4\pi}{3}\right) - \sin\left(\frac{4\pi}{3}\right) - i \cos\left(\frac{2\pi}{3}\right) + \sin\left(\frac{2\pi}{3}\right) \right], \\
    &= 2a_1 \cos\left(\frac{4\pi}{3}\right) - 2b_1 \sin\left(\frac{4\pi}{3}\right), \\
\end{align*}
}

\noindent
Using the approach of writing the model parameters in polar coordinates, the function, $S(\alpha)$ of all three states can be expressed as

\begin{align*}
    S(\alpha=0) &= r_1 e^{i \theta_1} \phi^1(\alpha=0) + r_2 e^{i \theta_2} \phi^2(\alpha=0),\\
    S(\alpha=1) &= r_1 e^{i \theta_1} \phi^1(\alpha=1) + r_2 e^{i \theta_2} \phi^2(\alpha=1),\\
    S(\alpha=2) &= r_1 e^{i \theta_1} \phi^1(\alpha=2) + r_2 e^{i \theta_2} \phi^2(\alpha=2).
\end{align*}

\noindent
Plugging in the values for the spin operators and using the constraints gives,

\begin{align*}
    S(\alpha=0) &= r_1 e^{i \theta_1} + r_2 e^{i \theta_2},\\
    &= r_1 e^{i \theta_1} + r_1 e^{-i \theta_1}, \\
    &= r_1 \left[ \cos\left( \theta_1 \right) + i \sin\left( \theta_1 \right) + \cos\left( \theta_1 \right) - i \sin\left( \theta_1 \right) \right],\\
    &= 2r_1 \cos\left( \theta_1 \right),\\
    S(\alpha=1) &= r_1 e^{i \theta_1} e^{\frac{2\pi i}{3}} + r_2 e^{i \theta_2} e^{\frac{4\pi i}{3}},\\
    &= r_1 e^{i\left( \frac{2\pi}{3} + \theta_1 \right)} + r_2 e^{-i \left(\frac{2\pi}{3} - \theta_2 \right)},\\
    &= r_1 e^{i\left( \frac{2\pi}{3} + \theta_1 \right)} + r_1 e^{-i \left(\frac{2\pi}{3} + \theta_1 \right)},\\
    &= r_1 \left[ \cos\left( \frac{2\pi}{3} + \theta_1 \right) + i \sin \left( \frac{2\pi}{3} + \theta_1 \right) + \cos\left( \frac{2\pi}{3} + \theta_1 \right) - i \sin \left( \frac{2\pi}{3} + \theta_1 \right) \right],\\
    &= 2 r_1 \cos\left( \frac{2\pi}{3} + \theta_1 \right),\\
    S(\alpha=2) &= r_1 e^{i \theta_1} e^{\frac{4\pi i}{3}} + r_2 e^{i \theta_2} e^{\frac{2\pi i}{3}},\\
    &= r_1 e^{i\left( \frac{4\pi}{3} + \theta_1 \right)} + r_2 e^{-i \left(\frac{4\pi}{3} - \theta_2 \right)},\\
    &= r_1 e^{i\left( \frac{4\pi}{3} + \theta_1 \right)} + r_1 e^{-i \left(\frac{4\pi}{3} + \theta_1 \right)},\\
    &= r_1 \left[ \cos\left( \frac{4\pi}{3} + \theta_1 \right) + i \sin \left( \frac{4\pi}{3} + \theta_1 \right) + \cos\left( \frac{4\pi}{3} + \theta_1 \right) - i \sin \left( \frac{4\pi}{3} + \theta_1 \right) \right],\\
    &= 2 r_1 \cos\left( \frac{4\pi}{3} + \theta_1 \right).
\end{align*}

\noindent 
From these expressions, we can see that $S(\alpha=0)$ is maximized when $\theta_1$ is a multiple of $2\pi$. $S(\alpha=1)$ is maximized when $\frac{2\pi}{3} + \theta_1$ is a multiple of $2\pi$ or equivalent when $\theta_1$ is equal to $-\frac{2\pi}{3}$.
$S(\alpha=2)$ is maximized when $\frac{4\pi}{3} + \theta_1$ is a multiple of $2\pi$ or equivalent when $\theta_1$ is equal to $\frac{-4\pi}{3}$.
In general, $S(\alpha)$ and thus the probability of state $\alpha$ is maximally increased by a pair of conjugate spin operators when the complex conjugate of the parameters are aligned with the spin operators applied on the state $\alpha$.

\noindent
In Figure \ref{fig:s_alpha}, a graphical representation of $S({\alpha})$ is shown for a 3-state,1-spin system where the complex conjugate of $g_1$ aligns mostly with $\phi^1(\alpha=1)$.
Due to the constraints on the parameters, the complex conjugate of $g_2$ aligns mostly with $\phi^2(\alpha=1)$.
Figure \ref{fig:s_alpha_1} shows that this alignment results in a large positive real contribution to $S(\alpha = 1)$ of both $g_1 \phi^1(\alpha=1)$ and $g_2 \phi^2(\alpha=1)$. 
Note that the complex contributions of $g_1 \phi^1(\alpha=1)$ and $g_2 \phi^2(\alpha=1)$ cancel each other, which makes $S(\alpha=1)$ a real value.
In Figure \ref{fig:s_alpha_2}, the same thing is shown for $S(\alpha=2)$. Because the action of the spin operator on the state is different, both spin operators will have a negative real contribution to $S(\alpha=2)$.
Again, the complex contributions cancel each other due to the constraints on the model parameters.

\begin{figure}[h]
    \centering
    \begin{subfigure}[b]{0.5\textwidth}
        \centering
        \resizebox{\textwidth}{!}{
        \begin{tikzpicture}[scale=2.7,cap=round,>=latex]
            %Origin
            \coordinate (O) at (0,0);
            %Coordinate (1,0)
            \coordinate (X) at (1,0);
            % draw the coordinates
            \draw[dashed] (-1.5cm,0cm) -- (1.5cm,0cm) node[right,fill=white] {};
            \draw[dashed] (0cm,-1.5cm) -- (0cm,1.5cm) node[above,fill=white] {};
    
            % draw the unit circle
            \draw[thick] (0cm,0cm) circle(1cm);
    
            %line for alpha = one state
            \draw[blue] (0cm,0cm) -- (120:1cm);
            % text at the end
            \draw (120:1.18cm) node[fill=white] {$\phi^1$};
            %draw angle
            \coordinate (phi1) at (120:1cm);
            \draw pic[->,"\footnotesize $\frac{2\pi}{3}$",draw=black,angle radius=30,angle eccentricity=1.25] {angle=X--O--phi1};
    
            %line for alpha = two state
            \draw[red] (0cm,0cm) -- (240:1cm);
            % text at the end
            \draw (240:1.18cm) node[fill=white] {$\phi^2$};
    
            %g1
            \draw[dashed, blue] (0cm,0cm) -- (260:1cm);
            % text at the end
            \draw (260:1.15cm) node[fill=white] {$g_1$};
            % draw angle
            \coordinate (G1) at (260:1cm);
            \draw pic[->,"\footnotesize $\theta_1$",draw=black,angle radius=20,angle eccentricity=1.3] {angle=X--O--G1};
    
            %g2
            \draw[dashed, red] (0cm,0cm) -- (100:1cm);
            % text at the end
            \draw (100:1.15cm) node[fill=white] {$g_2$};
    
            %g1 phi1
            \draw[dashdotted, blue] (0cm,0cm) -- (20:1cm);
            % dot at end point
            \filldraw[blue] (20:1cm) circle(0.6pt);
            % text at the end
            \draw (20:1.22cm) node[fill=white] {$g_1 \phi^1$};
            %draw angle
            \coordinate (G1_PHI1) at (20:1cm);
            \draw pic[->,"\footnotesize $\frac{2\pi}{3} +\theta_1$",draw=black,angle radius=40,angle eccentricity=1.5] {angle=X--O--G1_PHI1};
    
            %g2 phi2
            \draw[dashdotted, red] (0cm,0cm) -- (340:1cm);
            % dot at end point
            \filldraw[red] (340:1cm) circle(0.6pt);
            % text at the end
            \draw (340:1.22cm) node[fill=white] {$g_2 \phi^2$};
    
            % draw the horizontal and vertical coordinates
            \draw (1.4cm,0cm)  node[above=1pt] {$\Re$}
                (0cm,1.4cm)  node[fill=white] {$\Im$};
        \end{tikzpicture}}
        \caption{$S(\alpha = 1)$}\label{fig:s_alpha_1}
    \end{subfigure}%
    \begin{subfigure}[b]{0.5\textwidth}
        \centering
        \resizebox{\textwidth}{!}{
        \begin{tikzpicture}[scale=2.7,cap=round,>=latex]
            % draw the coordinates
            \draw[dashed] (-1.5cm,0cm) -- (1.5cm,0cm) node[right,fill=white] {};
            \draw[dashed] (0cm,-1.5cm) -- (0cm,1.5cm) node[above,fill=white] {};
    
            % draw the unit circle
            \draw[thick] (0cm,0cm) circle(1cm);

            %line for alpha = one state
            \draw[blue] (0cm,0cm) -- (240:1cm);
            % text at the end
            \draw (240:1.18cm) node[fill=white] {$\phi^1$};
    
            %line for alpha = two state
            \draw[red] (0cm,0cm) -- (120:1cm);
            % text at the end
            \draw (120:1.18cm) node[fill=white] {$\phi^2$};
    
            %g1
            \draw[dashed, blue] (0cm,0cm) -- (260:1cm);
            % text at the end
            \draw (260:1.15cm) node[fill=white] {$g_1$};
    
            %g2
            \draw[dashed, red] (0cm,0cm) -- (100:1cm);
            % text at the end
            \draw (100:1.15cm) node[fill=white] {$g_2$};
    
            %g1 phi1
            \draw[dashdotted, blue] (0cm,0cm) -- (140:1cm);
            % dot at end point
            \filldraw[blue] (140:1cm) circle(0.6pt);
            % text at the end
            \draw (140:1.22cm) node[fill=white] {$g_1 \phi^1$};
    
            %g2 phi2
            \draw[dashdotted, red] (0cm,0cm) -- (220:1cm);
            % dot at end point
            \filldraw[red] (220:1cm) circle(0.6pt);
            % text at the end
            \draw (220:1.22cm) node[fill=white] {$g_2 \phi^2$};

    
            % draw the horizontal and vertical coordinates
            \draw (1.4cm,0cm)  node[above=1pt] {$\Re$}
                (0cm,1.4cm)  node[fill=white] {$\Im$};
        \end{tikzpicture}}
        \caption{$S(\alpha = 2)$}\label{fig:s_alpha_2}
    \end{subfigure}
    \caption{Graphical representation of $S({\alpha})$ of a 3-state,1-spin system. The parameters $g_1$ and $g_2$ are chosen such that $g_1$ slightly deviates from the complex conjugate of $\phi^1(\alpha=1)$. The dots indicate the contributions to $S(\alpha)$.}
    \label{fig:s_alpha}
\end{figure}


\noindent
\textit{Case 1: align all model parameters with the complex conjugate of the corresponding spin operators all acting on one specific state ($\alpha = 2$)}

\begin{itemize}
    \item $g_1$ = $r\left[\cos\left( \frac{2\pi}{3}\right) + i \sin\left( \frac{2\pi}{3}\right)\right]$
    \item $g_2$ = $r\left[\cos\left( \frac{2\pi}{3}\right) - i \sin\left( \frac{2\pi}{3}\right)\right]$
\end{itemize}

\begin{figure}[h]
    \centering
    \begin{subfigure}[b]{0.33\textwidth}
        \centering
        \resizebox{\textwidth}{!}{
        \begin{tikzpicture}[scale=2.7,cap=round,>=latex]
            % draw the coordinates
            \draw[dashed] (-1.5cm,0cm) -- (1.5cm,0cm) node[right,fill=white] {};
            \draw[dashed] (0cm,-1.5cm) -- (0cm,1.5cm) node[above,fill=white] {};
    
            % draw the unit circle
            \draw[thick] (0cm,0cm) circle(1cm);
    
            % text at the end
            \draw (0:1.22cm) node[fill=white] {\begin{tabular}{c} $\phi^1$\\ $\phi^2$\\\end{tabular}};
    
            %line for alpha = one state
            \draw[blue] (0cm,0cm) -- (0:1cm);
    
            %line for alpha = two state
            \draw[red] (0cm,0cm) -- (0:1cm);

            %g1
            \draw[dashed, blue] (0cm,0cm) -- (120:1cm);    
            %g2
            \draw[dashed, red] (0cm,0cm) -- (240:1cm);

            %g1 phi1
            \draw[dashdotted, blue] (0cm,0cm) -- (120:1cm);
            % dot at end point
            \filldraw[blue] (120:1cm) circle(0.6pt);
            % text at the end
            \draw (120:1.3cm) node[fill=white] {\begin{tabular}{c} $g_1$\\$g_1 \phi^1$\end{tabular}};
    
            %g2 phi2
            \draw[dashdotted, red] (0cm,0cm) -- (240:1cm);
            % dot at end point
            \filldraw[red] (240:1cm) circle(0.6pt);
            % text at the end
            \draw (240:1.3cm) node[fill=white] {\begin{tabular}{c}$g_2$\\$g_2 \phi^2$\end{tabular}};
    
            % draw the horizontal and vertical coordinates
            \draw (1.4cm,0cm)  node[above=1pt] {$\Re$}
                (0cm,1.4cm)  node[fill=white] {$\Im$};
        \end{tikzpicture}}
        \caption{$S(\alpha = 0)$}\label{fig:case_1_s_alpha_0}
    \end{subfigure}%
    \begin{subfigure}[b]{0.33\textwidth}
        \centering
        \resizebox{\textwidth}{!}{
        \begin{tikzpicture}[scale=2.7,cap=round,>=latex]
            % draw the coordinates
            \draw[dashed] (-1.5cm,0cm) -- (1.5cm,0cm) node[right,fill=white] {};
            \draw[dashed] (0cm,-1.5cm) -- (0cm,1.5cm) node[above,fill=white] {};
    
            % draw the unit circle
            \draw[thick] (0cm,0cm) circle(1cm);
    
            %line for alpha = one state
            \draw[blue] (0cm,0cm) -- (120:1cm);
    
            %line for alpha = two state
            \draw[red] (0cm,0cm) -- (240:1cm);

            %g1
            \draw[dashed, blue] (0cm,0cm) -- (120:1cm);
            %g2
            \draw[dashed, red] (0cm,0cm) -- (240:1cm);
    
            %g1 phi1
            \draw[dashdotted, blue] (0cm,0cm) -- (240:1cm);
            % dot at end point
            \filldraw[blue] (240:1cm) circle(0.6pt);
            % text at the end
            \draw (240:1.4cm) node[fill=white] {\begin{tabular}{c}$g_2$\\$\phi^2$\\$g_1 \phi^1$ \end{tabular}};
    
            %g2 phi2
            \draw[dashdotted, red] (0cm,0cm) -- (120:1cm);
            % dot at end point
            \filldraw[red] (120:1cm) circle(0.6pt);
            % text at the end
            \draw (120:1.4cm) node[fill=white] {\begin{tabular}{c}$g_1$\\$\phi^1$\\$g_2 \phi^2$ \end{tabular}};
    
            % draw the horizontal and vertical coordinates
            \draw (1.4cm,0cm)  node[above=1pt] {$\Re$}
                (0cm,1.4cm)  node[fill=white] {$\Im$};
        \end{tikzpicture}}
        \caption{$S(\alpha = 1)$}\label{fig:case_1_s_alpha_1}
    \end{subfigure}%
    \begin{subfigure}[b]{0.33\textwidth}
        \centering
        \resizebox{\textwidth}{!}{
        \begin{tikzpicture}[scale=2.7,cap=round,>=latex]
            % draw the coordinates
            \draw[dashed] (-1.5cm,0cm) -- (1.5cm,0cm) node[right,fill=white] {};
            \draw[dashed] (0cm,-1.5cm) -- (0cm,1.5cm) node[above,fill=white] {};
    
            % draw the unit circle
            \draw[thick] (0cm,0cm) circle(1cm);

            % text at the end
            \draw (0:1.25cm) node[fill=white] {\begin{tabular}{c}$g_1 \phi^1$\\$g_2 \phi^2$ \end{tabular}};

            %line for alpha = one state
            \draw[blue] (0cm,0cm) -- (240:1cm);
            % text at the end
            \draw (240:1.3cm) node[fill=white] {\begin{tabular}{c}$g_2$\\$\phi^1$\end{tabular}};
    
            %line for alpha = two state
            \draw[red] (0cm,0cm) -- (120:1cm);
            % text at the end
            \draw (120:1.3cm) node[fill=white] {\begin{tabular}{c}$g_1$\\$\phi^2$\end{tabular}};
    
            %g1
            \draw[dashed, blue] (0cm,0cm) -- (120:1cm);
            %g2
            \draw[dashed, red] (0cm,0cm) -- (240:1cm);

            %g1 phi1
            \draw[dashdotted, blue] (0cm,0cm) -- (0:1cm);
            % dot at end point
            \filldraw[blue] (0:1cm) circle(0.8pt);
    
            %g2 phi2
            \draw[dashdotted, red] (0cm,0cm) -- (0:1cm);
            % dot at end point
            \filldraw[red] (0:1cm) circle(0.4pt);
    
            % draw the horizontal and vertical coordinates
            \draw (1.4cm,0cm)  node[above=1pt] {$\Re$}
                (0cm,1.4cm)  node[fill=white] {$\Im$};
        \end{tikzpicture}}
        \caption{$S(\alpha = 2)$}\label{fig:case_1_s_alpha_2}
    \end{subfigure}
    \caption{Graphical representation of $S({\alpha})$ of a 3-state,1-spin system. The parameters $g_1$ and $g_2$ are chosen such that all of them align with the complex conjugate of corresponding spin operator acting on $\alpha=2$.}
    \label{fig:s_alpha}
\end{figure}

\begin{table}[h]
    \centering
    \caption{Numerical values for S($\alpha$) and P($\alpha$).}
    \label{tab:case_1_num_values}
    \begin{subtable}{.3\textwidth}
        \centering
        \caption{r = 0.5}
        \begin{tabular}{ccc}
            \toprule
             $\alpha$ & S($\alpha$) & P($\alpha$)\\
            \midrule
            0 & -0.5 & 0.154 \\
            1 & -0.5 & 0.154 \\
            2 & 1 & 0.691 \\
          \bottomrule
        \end{tabular}
    \end{subtable}%
    \begin{subtable}{.3\textwidth}
        \centering
        \caption{r = 1}
        \begin{tabular}{ccc}
            \toprule
             $\alpha$ & S($\alpha$) & P($\alpha$)\\
            \midrule
            0 & -1 & 0.045 \\
            1 & -1 & 0.045 \\
            2 & 2 & 0.909 \\
          \bottomrule
        \end{tabular}
    \end{subtable}%
    \begin{subtable}{.3\textwidth}
        \centering
        \caption{r = 2}
        \begin{tabular}{ccc}
            \toprule
            $\alpha$ & S($\alpha$) & P($\alpha$)\\
            \midrule
            0 & -2 & 0.002 \\
            1 & -2 & 0.002 \\
            2 & 4 & 0.995 \\
            \bottomrule
        \end{tabular}
    \end{subtable}
\end{table}

\noindent
In the graphical representation, the magnitude (r) of all interaction parameters is equal to one. Due to the alignment of the interaction parameters, there is preference for state $\alpha = 2$.
Increasing the magnitude of the interaction parameters, increases this preference and vice versa as shown in Table \ref{tab:case_1_num_values}. 

\subsubsection{Example: 3-state,2-spin system}

\noindent
\textit{Case 1: Single second-order interaction}

\begin{itemize}
    \item $g_{11}$ = $r\left[\cos\left( \frac{4\pi}{3}\right) + i \sin\left( \frac{4\pi}{3}\right)\right]$
    \item $g_{22}$ = $r\left[\cos\left( \frac{4\pi}{3}\right) - i \sin\left( \frac{4\pi}{3}\right)\right]$
\end{itemize}

\begin{figure}[h]
    \centering
    \begin{subfigure}[b]{0.33\textwidth}
        \centering
        \resizebox{\textwidth}{!}{
        \begin{tikzpicture}[scale=2.7,cap=round,>=latex]
            % draw the coordinates
            \draw[dashed] (-1.5cm,0cm) -- (1.5cm,0cm) node[right,fill=white] {};
            \draw[dashed] (0cm,-1.5cm) -- (0cm,1.5cm) node[above,fill=white] {};
    
            % draw the unit circle
            \draw[thick] (0cm,0cm) circle(1cm);
    
            % text at the end of 0 angle
            \draw (0:1.22cm) node[fill=white] {\begin{tabular}{c} $\phi^{11}$\\ $\phi^{22}$ \end{tabular}};
    
            %phi11
            \draw[blue] (0cm,0cm) -- (0:1cm);
    
            %phi22
            \draw[red] (0cm,0cm) -- (0:1cm);

            %g11
            \draw[dashed, blue] (0cm,0cm) -- (240:1cm);
            % text at the end
    
            %g22
            \draw[dashed, red] (0cm,0cm) -- (120:1cm);
            % text at the end

            %g11 phi11
            \draw[dashdotted, blue] (0cm,0cm) -- (240:1cm);
            % dot at end point
            \filldraw[blue] (240:1cm) circle(0.6pt);
            % text at the end
            \draw (240:1.32cm) node[fill=white] {\begin{tabular}{c} $g_{11}$\\$g_{11} \phi^{11}$\end{tabular}};
    
            %g22 phi22
            \draw[dashdotted, red] (0cm,0cm) -- (120:1cm);
            % dot at end point
            \filldraw[red] (120:1cm) circle(0.6pt);
            % text at the end
            \draw (120:1.32cm) node[fill=white] {\begin{tabular}{c}$g_{22}$\\$g_{22} \phi^{22}$\end{tabular}};
    
            % draw the horizontal and vertical coordinates
            \draw (1.4cm,0cm)  node[above=1pt] {}
                (0cm,1.4cm)  node[fill=white] {$\Im$};
        \end{tikzpicture}}
        \caption{$S(\boldsymbol{\alpha} = \{00, 12, 21\})$}\label{fig:case_1_s_alpha_00_g11}
    \end{subfigure}%
    \begin{subfigure}[b]{0.33\textwidth}
        \centering
        \resizebox{\textwidth}{!}{
        \begin{tikzpicture}[scale=2.7,cap=round,>=latex]
            % draw the coordinates
            \draw[dashed] (-1.5cm,0cm) -- (1.5cm,0cm) node[right,fill=white] {};
            \draw[dashed] (0cm,-1.5cm) -- (0cm,1.5cm) node[above,fill=white] {};
    
            % draw the unit circle
            \draw[thick] (0cm,0cm) circle(1cm);
    
            % text at the end of the 0 angle
            \draw (0:1.3cm) node[fill=white] {\begin{tabular}{c} $g_{11}\phi^{11}$\\ $g_{22}\phi^{22}$ \end{tabular}};
    
            %phi11
            \draw[blue] (0cm,0cm) -- (120:1cm);
            \draw (120:1.3cm) node[fill=white] {\begin{tabular}{c} $g_{22}$\\$\phi^{11}$\end{tabular}};
    
            %phi22
            \draw[red] (0cm,0cm) -- (240:1cm);
            \draw (240:1.3cm) node[fill=white] {\begin{tabular}{c} $g_{11}$\\$\phi^{22}$\end{tabular}};

            %g11
            \draw[dashed, blue] (0cm,0cm) -- (240:1cm);
    
            %g22
            \draw[dashed, red] (0cm,0cm) -- (120:1cm);
    
            %g11 phi11
            \draw[dashdotted, blue] (0cm,0cm) -- (0:1cm);
            % dot at end point
            \filldraw[blue] (0:1cm) circle(0.8pt);
    
            %g22 phi22
            \draw[dashdotted, red] (0cm,0cm) -- (0:1cm);
            % dot at end point
            \filldraw[red] (0:1cm) circle(0.4pt);
    
            % draw the horizontal and vertical coordinates
            \draw (1.4cm,0cm)  node[above=1pt] {}
                (0cm,1.4cm)  node[fill=white] {$\Im$};
        \end{tikzpicture}}
        \caption{$S(\boldsymbol{\alpha} = \{01, 10, 22\})$}\label{fig:case_1_s_alpha_01_g11}
    \end{subfigure}%
    \begin{subfigure}[b]{0.33\textwidth}
        \centering
        \resizebox{\textwidth}{!}{
        \begin{tikzpicture}[scale=2.7,cap=round,>=latex]
            % draw the coordinates
            \draw[dashed] (-1.5cm,0cm) -- (1.5cm,0cm) node[right,fill=white] {};
            \draw[dashed] (0cm,-1.5cm) -- (0cm,1.5cm) node[above,fill=white] {};
    
            % draw the unit circle
            \draw[thick] (0cm,0cm) circle(1cm);

            %phi11
            \draw[blue] (0cm,0cm) -- (240:1cm);
            % text at the end
            \draw (240:1.4cm) node[fill=white] {\begin{tabular}{c}$g_{11}$\\$\phi^{11}$\\$g_{22}\phi^{22}$\end{tabular}};
    
            %phi22
            \draw[red] (0cm,0cm) -- (120:1cm);
            % text at the end
            \draw (120:1.4cm) node[fill=white] {\begin{tabular}{c}$g_{22}$\\$\phi^{22}$\\$g_{11}\phi^{11}$\end{tabular}};
    
            %g11
            \draw[dashed, blue] (0cm,0cm) -- (240:1cm);
    
            %g22
            \draw[dashed, red] (0cm,0cm) -- (120:1cm);

            %g11 phi11
            \draw[dashdotted, blue] (0cm,0cm) -- (120:1cm);
            % dot at end point
            \filldraw[blue] (120:1cm) circle(0.6pt);
    
            %g22 phi22
            \draw[dashdotted, red] (0cm,0cm) -- (240:1cm);
            % dot at end point
            \filldraw[red] (240:1cm) circle(0.6pt);

    
            % draw the horizontal and vertical coordinates
            \draw (1.4cm,0cm)  node[above=1pt] {}
                (0cm,1.4cm)  node[fill=white] {$\Im$};
        \end{tikzpicture}}
        \caption{$S(\boldsymbol{\alpha} = \{02, 11, 20\})$}\label{fig:case_1_s_alpha_02_g11}
    \end{subfigure}
    \caption{Graphical representation of $S(\boldsymbol{\alpha})$ of a 3-state,2-spin system. The parameter $g_{11}$ is chosen such that it aligns with the complex conjugate of corresponding spin operator acting on states for which $\sum_{j \in \mu} \alpha_j \mu_j =  1\mod3$.}
    \label{fig:case_1_s_alpha_g11}
\end{figure}

\begin{table}[h]
    \centering
    \caption{Numerical values for S($\boldsymbol{\alpha}$) and P($\boldsymbol{\alpha}$) for a spin model with only $\phi^{11}$ and $\phi^{22}$.}
    \label{tab:case_1_num_values_g11}
    \begin{subtable}{.3\textwidth}
        \centering
        \caption{r = 0.5}
        \begin{tabular}{ccc}
            \toprule
             $\boldsymbol{\alpha}$ & S($\boldsymbol{\alpha}$) & P($\boldsymbol{\alpha}$)\\
            \midrule
            00 & -0.5 & 0.051 \\
            01 & 1 & 0.230 \\
            02 & -0.5 & 0.051 \\
            10 & 1 & 0.230 \\
            11 & -0.5 & 0.051 \\
            12 & -0.5 & 0.051 \\
            20 & -0.5 & 0.051 \\
            21 & -0.5 & 0.051 \\
            22 & 1 & 0.230 \\
          \bottomrule
        \end{tabular}
    \end{subtable}%
    \begin{subtable}{.3\textwidth}
        \centering
        \caption{r = 1}
        \begin{tabular}{ccc}
            \toprule
             $\boldsymbol{\alpha}$ & S($\boldsymbol{\alpha}$) & P($\boldsymbol{\alpha}$)\\
            \midrule
            00 & -1 & 0.015 \\
            01 & 2 & 0.303 \\
            02 & -1 & 0.015 \\
            10 & 2 & 0.303 \\
            11 & -1 & 0.015 \\
            12 & -1 & 0.015 \\
            20 & -1 & 0.015 \\
            21 & -1 & 0.015 \\
            22 & 2 & 0.303 \\
          \bottomrule
        \end{tabular}
    \end{subtable}%
    \begin{subtable}{.3\textwidth}
        \centering
        \caption{r = 2}
        \begin{tabular}{ccc}
            \toprule
             $\boldsymbol{\alpha}$ & S($\boldsymbol{\alpha}$) & P($\boldsymbol{\alpha}$)\\
            \midrule
            00 & -2 & 0.001 \\
            01 & 4 & 0.332 \\
            02 & -2 & 0.001 \\
            10 & 4 & 0.332 \\
            11 & -2 & 0.001 \\
            12 & -2 & 0.001 \\
            20 & -2 & 0.001 \\
            21 & -2 & 0.001 \\
            22 & 4 & 0.332 \\
          \bottomrule
        \end{tabular}
    \end{subtable}
\end{table}


\begin{itemize}
    \item $g_{12}$ = $r\left[\cos\left( \frac{4\pi}{3}\right) + i \sin\left( \frac{4\pi}{3}\right)\right]$
    \item $g_{21}$ = $r\left[\cos\left( \frac{4\pi}{3}\right) - i \sin\left( \frac{4\pi}{3}\right)\right]$
\end{itemize}

\begin{figure}[h]
    \centering
    \begin{subfigure}[b]{0.33\textwidth}
        \centering
        \resizebox{\textwidth}{!}{
        \begin{tikzpicture}[scale=2.7,cap=round,>=latex]
            % draw the coordinates
            \draw[dashed] (-1.5cm,0cm) -- (1.5cm,0cm) node[right,fill=white] {};
            \draw[dashed] (0cm,-1.5cm) -- (0cm,1.5cm) node[above,fill=white] {};
    
            % draw the unit circle
            \draw[thick] (0cm,0cm) circle(1cm);
    
            % text at the end of 0 angle
            \draw (0:1.22cm) node[fill=white] {\begin{tabular}{c} $\phi^{12}$\\ $\phi^{21}$ \end{tabular}};
    
            %phi12
            \draw[blue] (0cm,0cm) -- (0:1cm);
    
            %phi21
            \draw[red] (0cm,0cm) -- (0:1cm);

            %g12
            \draw[dashed, blue] (0cm,0cm) -- (240:1cm);
            % text at the end
    
            %g21
            \draw[dashed, red] (0cm,0cm) -- (120:1cm);
            % text at the end

            %g12 phi12
            \draw[dashdotted, blue] (0cm,0cm) -- (240:1cm);
            % dot at end point
            \filldraw[blue] (240:1cm) circle(0.6pt);
            % text at the end
            \draw (240:1.32cm) node[fill=white] {\begin{tabular}{c} $g_{12}$\\$g_{12} \phi^{12}$\end{tabular}};
    
            %g21 phi21
            \draw[dashdotted, red] (0cm,0cm) -- (120:1cm);
            % dot at end point
            \filldraw[red] (120:1cm) circle(0.6pt);
            % text at the end
            \draw (120:1.32cm) node[fill=white] {\begin{tabular}{c}$g_{21}$\\$g_{21} \phi^{21}$\end{tabular}};
    
            % draw the horizontal and vertical coordinates
            \draw (1.4cm,0cm)  node[above=1pt] {}
                (0cm,1.4cm)  node[fill=white] {$\Im$};
        \end{tikzpicture}}
        \caption{$S(\boldsymbol{\alpha} = \{00, 11, 22\})$}\label{fig:case_1_s_alpha_00_g12}
    \end{subfigure}%
    \begin{subfigure}[b]{0.33\textwidth}
        \centering
        \resizebox{\textwidth}{!}{
            \begin{tikzpicture}[scale=2.7,cap=round,>=latex]
                % draw the coordinates
                \draw[dashed] (-1.5cm,0cm) -- (1.5cm,0cm) node[right,fill=white] {};
                \draw[dashed] (0cm,-1.5cm) -- (0cm,1.5cm) node[above,fill=white] {};
        
                % draw the unit circle
                \draw[thick] (0cm,0cm) circle(1cm);
    
                %phi11
                \draw[blue] (0cm,0cm) -- (240:1cm);
                % text at the end
                \draw (240:1.4cm) node[fill=white] {\begin{tabular}{c}$g_{12}$\\$\phi^{12}$\\$g_{21}\phi^{21}$\end{tabular}};
        
                %phi22
                \draw[red] (0cm,0cm) -- (120:1cm);
                % text at the end
                \draw (120:1.4cm) node[fill=white] {\begin{tabular}{c}$g_{21}$\\$\phi^{21}$\\$g_{12}\phi^{12}$\end{tabular}};
        
                %g11
                \draw[dashed, blue] (0cm,0cm) -- (240:1cm);
        
                %g22
                \draw[dashed, red] (0cm,0cm) -- (120:1cm);
    
                %g11 phi11
                \draw[dashdotted, blue] (0cm,0cm) -- (120:1cm);
                % dot at end point
                \filldraw[blue] (120:1cm) circle(0.6pt);
        
                %g22 phi22
                \draw[dashdotted, red] (0cm,0cm) -- (240:1cm);
                % dot at end point
                \filldraw[red] (240:1cm) circle(0.6pt);
    
        
                % draw the horizontal and vertical coordinates
                \draw (1.4cm,0cm)  node[above=1pt] {}
                    (0cm,1.4cm)  node[fill=white] {$\Im$};
            \end{tikzpicture}}
        \caption{$S(\boldsymbol{\alpha} = \{01, 12, 20\})$}\label{fig:case_1_s_alpha_01_g12}
    \end{subfigure}%
    \begin{subfigure}[b]{0.33\textwidth}
        \centering
        \resizebox{\textwidth}{!}{
            \begin{tikzpicture}[scale=2.7,cap=round,>=latex]
                % draw the coordinates
                \draw[dashed] (-1.5cm,0cm) -- (1.5cm,0cm) node[right,fill=white] {};
                \draw[dashed] (0cm,-1.5cm) -- (0cm,1.5cm) node[above,fill=white] {};
        
                % draw the unit circle
                \draw[thick] (0cm,0cm) circle(1cm);
        
                % text at the end of the 0 angle
                \draw (0:1.32cm) node[fill=white] {\hspace{-.3cm} \begin{tabular}{c} $g_{12}\phi^{12}$\\ $g_{21}\phi^{21}$ \end{tabular}};
        
                %phi11
                \draw[blue] (0cm,0cm) -- (120:1cm);
                \draw (120:1.3cm) node[fill=white] {\begin{tabular}{c} $g_{21}$\\$\phi^{12}$\end{tabular}};
        
                %phi22
                \draw[red] (0cm,0cm) -- (240:1cm);
                \draw (240:1.3cm) node[fill=white] {\begin{tabular}{c} $g_{12}$\\$\phi^{21}$\end{tabular}};
    
                %g11
                \draw[dashed, blue] (0cm,0cm) -- (240:1cm);
        
                %g22
                \draw[dashed, red] (0cm,0cm) -- (120:1cm);
        
                %g11 phi11
                \draw[dashdotted, blue] (0cm,0cm) -- (0:1cm);
                % dot at end point
                \filldraw[blue] (0:1cm) circle(0.8pt);
        
                %g22 phi22
                \draw[dashdotted, red] (0cm,0cm) -- (0:1cm);
                % dot at end point
                \filldraw[red] (0:1cm) circle(0.4pt);
        
                % draw the horizontal and vertical coordinates
                \draw (1.5cm,0cm)  node[above=1pt] {}
                    (0cm,1.4cm)  node[fill=white] {$\Im$};
            \end{tikzpicture}}
        \caption{$S(\boldsymbol{\alpha} = \{02, 10, 21\})$}\label{fig:case_1_s_alpha_02_g12}
    \end{subfigure}
    \caption{Graphical representation of $S({\boldsymbol{\alpha}})$ of a 3-state,2-spin system. The parameter $g_{12}$ is chosen such that it aligns with the complex conjugate of corresponding spin operator acting on states for which $\sum_{j \in \mu} \alpha_j \mu_j =  1\mod3$.}
    \label{fig:s_case_1_alpha_g12}
\end{figure}

\begin{table}[h]
    \centering
    \caption{Numerical values for S($\boldsymbol{\alpha}$) and P($\boldsymbol{\alpha}$) for a spin model with only $\phi^{12}$ and $\phi^{21}$.}
    \label{tab:case_1_num_values_g11}
    \begin{subtable}{.3\textwidth}
        \centering
        \caption{r = 0.5}
        \begin{tabular}{ccc}
            \toprule
             $\boldsymbol{\alpha}$ & S($\boldsymbol{\alpha}$) & P($\boldsymbol{\alpha}$)\\
            \midrule
            00 & -0.5 & 0.051 \\
            01 & -0.5 & 0.051 \\
            02 & 1 & 0.230 \\
            10 & 1 & 0.230 \\
            11 & -0.5 & 0.051 \\
            12 & -0.5 & 0.051 \\
            20 & -0.5 & 0.051 \\
            21 & 1 & 0.230 \\
            22 & -0.5 & 0.051\\
          \bottomrule
        \end{tabular}
    \end{subtable}%
    \begin{subtable}{.3\textwidth}
        \centering
        \caption{r = 1}
        \begin{tabular}{ccc}
            \toprule
             $\boldsymbol{\alpha}$ & S($\boldsymbol{\alpha}$) & P($\boldsymbol{\alpha}$)\\
            \midrule
            00 & -1 & 0.015 \\
            01 & -1 & 0.015 \\
            02 & 2 & 0.303 \\
            10 & 2 & 0.303 \\
            11 & -1 & 0.015 \\
            12 & -1 & 0.015 \\
            20 & -1 & 0.015 \\
            21 & 2 & 0.303 \\
            22 & -1 & 0.015 \\
          \bottomrule
        \end{tabular}
    \end{subtable}%
    \begin{subtable}{.3\textwidth}
        \centering
        \caption{r = 2}
        \begin{tabular}{ccc}
            \toprule
             $\boldsymbol{\alpha}$ & S($\boldsymbol{\alpha}$) & P($\boldsymbol{\alpha}$)\\
            \midrule
            00 & -2 & 0.001 \\
            01 & -2 & 0.001 \\
            02 & 4 & 0.332 \\
            10 & 4 & 0.332 \\
            11 & -2 & 0.001 \\
            12 & -2 & 0.001 \\
            20 & -2 & 0.001 \\
            21 & 4 & 0.332 \\
            22 & -2 & 0.001 \\
          \bottomrule
        \end{tabular}
    \end{subtable}
\end{table}

\newpage

\noindent
\textit{Case 2: Two first-order interactions}

\begin{itemize}
    \item $g_{01}$ = $r\left[\cos\left( \frac{2\pi}{3}\right) + i \sin\left( \frac{2\pi}{3}\right)\right]$
    \item $g_{02}$ = $r\left[\cos\left( \frac{2\pi}{3}\right) - i \sin\left( \frac{2\pi}{3}\right)\right]$
    \item $g_{10}$ = $r\left[\cos\left( \frac{4\pi}{3}\right) + i \sin\left( \frac{4\pi}{3}\right)\right]$
    \item $g_{20}$ = $r\left[\cos\left( \frac{4\pi}{3}\right) - i \sin\left( \frac{4\pi}{3}\right)\right]$
\end{itemize}

\begin{figure}[h!]
    \centering
    \begin{subfigure}[b]{0.33\textwidth}
        \centering
        \resizebox{\textwidth}{!}{
        \begin{tikzpicture}[scale=2.7,cap=round,>=latex]
            % draw the coordinates
            \draw[dashed] (-1.5cm,0cm) -- (1.5cm,0cm) node[right,fill=white] {};
            \draw[dashed] (0cm,-1.5cm) -- (0cm,1.5cm) node[above,fill=white] {};
    
            % draw the unit circle
            \draw[thick] (0cm,0cm) circle(1cm);

            %g01 phi01
            \draw[black] (0cm,0cm) -- (120:1cm);
            % dot at end point
            \filldraw[black] (120:1cm) circle(0.8pt);
            %g20 phi20
            \draw[black] (0cm,0cm) -- (120:1cm);
            % dot at end point
            \filldraw[black] (120:1cm) circle(0.4pt);

            %g02 phi02
            \draw[black] (0cm,0cm) -- (240:1cm);
            % dot at end point
            \filldraw[black] (240:1cm) circle(0.8pt);
            %g10 phi10
            \draw[black] (0cm,0cm) -- (240:1cm);
            % dot at end point
            \filldraw[black] (240:1cm) circle(0.4pt);

            % text at the end
            \draw (120:1.32cm) node[fill=white] {\begin{tabular}{c} $g_{01} \phi^{01}$\\ $g_{20} \phi^{20}$\end{tabular}};
            \draw (240:1.32cm) node[fill=white] {\begin{tabular}{c} $g_{02} \phi^{02}$\\ $g_{10} \phi^{10}$\end{tabular}};
    
            % draw the horizontal and vertical coordinates
            \draw (1.4cm,0cm)  node[above=1pt] {}
                (0cm,1.4cm)  node[fill=white] {$\Im$};
        \end{tikzpicture}}
        \caption{$S(\boldsymbol{\alpha} = 00)$}\label{fig:case_2_s_alpha_00}
    \end{subfigure}%
    \begin{subfigure}[b]{0.33\textwidth}
        \centering
        \resizebox{\textwidth}{!}{
            \begin{tikzpicture}[scale=2.7,cap=round,>=latex]
                % draw the coordinates
                \draw[dashed] (-1.5cm,0cm) -- (1.5cm,0cm) node[right,fill=white] {};
                \draw[dashed] (0cm,-1.5cm) -- (0cm,1.5cm) node[above,fill=white] {};
        
                % draw the unit circle
                \draw[thick] (0cm,0cm) circle(1cm);
    
                %g02 phi02
                \draw[black] (0cm,0cm) -- (120:1cm);
                % dot at end point
                \filldraw[black] (120:1cm) circle(0.8pt);
                %g20 phi20
                \draw[black] (0cm,0cm) -- (120:1cm);
                % dot at end point
                \filldraw[black] (120:1cm) circle(0.4pt);                
    
                %g01 phi01
                \draw[black] (0cm,0cm) -- (240:1cm);
                % dot at end point
                \filldraw[black] (240:1cm) circle(0.8pt);
                %g10 phi10
                \draw[black] (0cm,0cm) -- (240:1cm);
                % dot at end point
                \filldraw[black] (240:1cm) circle(0.4pt);

                % text at the end
                \draw (120:1.32cm) node[fill=white] {\begin{tabular}{c} $g_{02} \phi^{02}$\\ $g_{20} \phi^{20}$\end{tabular}};
                \draw (240:1.32cm) node[fill=white] {\begin{tabular}{c} $g_{01} \phi^{01}$\\ $g_{10} \phi^{10}$\end{tabular}};
        
                % draw the horizontal and vertical coordinates
                \draw (1.4cm,0cm)  node[above=1pt] {}
                    (0cm,1.4cm)  node[fill=white] {$\Im$};
            \end{tikzpicture}}
            \caption{$S(\boldsymbol{\alpha} = 01)$}\label{fig:case_2_s_alpha_01}
    \end{subfigure}%
    \begin{subfigure}[b]{0.33\textwidth}
        \centering
        \resizebox{\textwidth}{!}{
            \begin{tikzpicture}[scale=2.7,cap=round,>=latex]
                % draw the coordinates
                \draw[dashed] (-1.5cm,0cm) -- (1.5cm,0cm) node[right,fill=white] {};
                \draw[dashed] (0cm,-1.5cm) -- (0cm,1.5cm) node[above,fill=white] {};
        
                % draw the unit circle
                \draw[thick] (0cm,0cm) circle(1cm);
    
                %g01 phi01
                \draw[black] (0cm,0cm) -- (0:1cm);
                % dot at end point
                \filldraw[black] (0:1cm) circle(0.8pt);
                
                %g20 phi20
                \draw[black] (0cm,0cm) -- (120:1cm);
                % dot at end point
                \filldraw[black] (120:1cm) circle(0.6pt);               
    
                %g02 phi02
                \draw[black] (0cm,0cm) -- (240:1cm);
                % dot at end point
                \filldraw[black] (240:1cm) circle(0.4pt);

                %g10 phi10
                \draw[black] (0cm,0cm) -- (240:1cm);
                % dot at end point
                \filldraw[black] (240:1cm) circle(0.6pt);

                % text at the end
                \draw (0:1.32cm) node[fill=white] {\hspace{-.3cm} \begin{tabular}{c}$g_{01} \phi^{01}$\\ $g_{02} \phi^{02}$\end{tabular}};
                \draw (120:1.25cm) node[fill=white] {\begin{tabular}{c} $g_{20} \phi^{20}$\end{tabular}};
                \draw (240:1.25cm) node[fill=white] {\begin{tabular}{c} $g_{10} \phi^{10}$\end{tabular}};
        
                % draw the horizontal and vertical coordinates
                \draw (1.4cm,0cm)  node[above=1pt] {}
                    (0cm,1.4cm)  node[fill=white] {$\Im$};
            \end{tikzpicture}}
            \caption{$S(\boldsymbol{\alpha} = 02)$}\label{fig:case_2_s_alpha_02}
    \end{subfigure}
    \begin{subfigure}[b]{0.33\textwidth}
        \centering
        \resizebox{\textwidth}{!}{
        \begin{tikzpicture}[scale=2.7,cap=round,>=latex]
            % draw the coordinates
            \draw[dashed] (-1.5cm,0cm) -- (1.5cm,0cm) node[right,fill=white] {};
            \draw[dashed] (0cm,-1.5cm) -- (0cm,1.5cm) node[above,fill=white] {};
    
            % draw the unit circle
            \draw[thick] (0cm,0cm) circle(1cm);

            %g01 phi01
            \draw[black] (0cm,0cm) -- (120:1cm);
            % dot at end point
            \filldraw[black] (120:1cm) circle(0.6pt);
            %g20 phi20
            \draw[black] (0cm,0cm) -- (0:1cm);
            % dot at end point
            \filldraw[black] (0:1cm) circle(0.8pt);

            %g02 phi02
            \draw[black] (0cm,0cm) -- (240:1cm);
            % dot at end point
            \filldraw[black] (240:1cm) circle(0.6pt);
            %g10 phi10
            \draw[black] (0cm,0cm) -- (0:1cm);
            % dot at end point
            \filldraw[black] (0:1cm) circle(0.4pt);

            % text at the end
            \draw (0:1.32cm) node[fill=white] {\hspace{-.3cm} \begin{tabular}{c} $g_{10} \phi^{10}$\\ $g_{20} \phi^{20}$\end{tabular}};
            \draw (120:1.32cm) node[fill=white] {\begin{tabular}{c} $g_{01} \phi^{01}$\end{tabular}};
            \draw (240:1.32cm) node[fill=white] {\begin{tabular}{c} $g_{02} \phi^{02}$\end{tabular}};
    
            % draw the horizontal and vertical coordinates
            \draw (1.4cm,0cm)  node[above=1pt] {}
                (0cm,1.4cm)  node[fill=white] {$\Im$};
        \end{tikzpicture}}
        \caption{$S(\boldsymbol{\alpha} = 10)$}\label{fig:case_2_s_alpha_10}
    \end{subfigure}%
    \begin{subfigure}[b]{0.33\textwidth}
        \centering
        \resizebox{\textwidth}{!}{
            \begin{tikzpicture}[scale=2.7,cap=round,>=latex]
                % draw the coordinates
                \draw[dashed] (-1.5cm,0cm) -- (1.5cm,0cm) node[right,fill=white] {};
                \draw[dashed] (0cm,-1.5cm) -- (0cm,1.5cm) node[above,fill=white] {};
        
                % draw the unit circle
                \draw[thick] (0cm,0cm) circle(1cm);
    
                %g02 phi02
                \draw[black] (0cm,0cm) -- (120:1cm);
                % dot at end point
                \filldraw[black] (120:1cm) circle(0.6pt);
                %g20 phi20
                \draw[black] (0cm,0cm) -- (0:1cm);
                % dot at end point
                \filldraw[black] (0:1cm) circle(0.8pt);                
    
                %g01 phi01
                \draw[black] (0cm,0cm) -- (240:1cm);
                % dot at end point
                \filldraw[black] (240:1cm) circle(0.6pt);
                %g10 phi10
                \draw[black] (0cm,0cm) -- (0:1cm);
                % dot at end point
                \filldraw[black] (0:1cm) circle(0.4pt);

                % text at the end
                \draw (0:1.32cm) node[fill=white] {\hspace{-.3cm} \begin{tabular}{c} $g_{10} \phi^{10}$\\ $g_{20} \phi^{20}$\end{tabular}};
                \draw (120:1.32cm) node[fill=white] {\begin{tabular}{c} $g_{02} \phi^{02}$\end{tabular}};
                \draw (240:1.32cm) node[fill=white] {\begin{tabular}{c} $g_{01} \phi^{01}$\end{tabular}};
        
                % draw the horizontal and vertical coordinates
                \draw (1.4cm,0cm)  node[above=1pt] {}
                    (0cm,1.4cm)  node[fill=white] {$\Im$};
            \end{tikzpicture}}
            \caption{$S(\boldsymbol{\alpha} = 11)$}\label{fig:case_2_s_alpha_11}
    \end{subfigure}%
    \begin{subfigure}[b]{0.33\textwidth}
        \centering
        \resizebox{\textwidth}{!}{
            \begin{tikzpicture}[scale=2.7,cap=round,>=latex]
                % draw the coordinates
                \draw[dashed] (-1.5cm,0cm) -- (1.5cm,0cm) node[right,fill=white] {};
                \draw[dashed] (0cm,-1.5cm) -- (0cm,1.5cm) node[above,fill=white] {};
        
                % draw the unit circle
                \draw[thick] (0cm,0cm) circle(1cm);
    
                %g01 phi01
                \draw[black] (0cm,0cm) -- (0:1cm);
                % dot at end point
                \filldraw[black] (0:1cm) circle(1pt);
                
                %g20 phi20
                \draw[black] (0cm,0cm) -- (0:1cm);
                % dot at end point
                \filldraw[black] (0:1cm) circle(0.8pt);               
    
                %g02 phi02
                \draw[black] (0cm,0cm) -- (0:1cm);
                % dot at end point
                \filldraw[black] (0:1cm) circle(0.6pt);

                %g10 phi10
                \draw[black] (0cm,0cm) -- (0:1cm);
                % dot at end point
                \filldraw[black] (0:1cm) circle(0.4pt);

                % text at the end
                \draw (0:1.32cm) node[fill=white] {\hspace{-.3cm} \begin{tabular}{c}$g_{01} \phi^{01}$\\ $g_{02} \phi^{02}$\\$g_{10} \phi^{10}$\\ $g_{20} \phi^{20}$\end{tabular}};
        
                % draw the horizontal and vertical coordinates
                \draw (1.4cm,0cm)  node[above=1pt] {}
                    (0cm,1.4cm)  node[fill=white] {$\Im$};
            \end{tikzpicture}}
            \caption{$S(\boldsymbol{\alpha} = 12)$}\label{fig:case_2_s_alpha_12}
    \end{subfigure}
    \begin{subfigure}[b]{0.33\textwidth}
        \centering
        \resizebox{\textwidth}{!}{
        \begin{tikzpicture}[scale=2.7,cap=round,>=latex]
            % draw the coordinates
            \draw[dashed] (-1.5cm,0cm) -- (1.5cm,0cm) node[right,fill=white] {};
            \draw[dashed] (0cm,-1.5cm) -- (0cm,1.5cm) node[above,fill=white] {};
    
            % draw the unit circle
            \draw[thick] (0cm,0cm) circle(1cm);

            %g01 phi01
            \draw[black] (0cm,0cm) -- (120:1cm);
            % dot at end point
            \filldraw[black] (120:1cm) circle(0.8pt);
            %g20 phi20
            \draw[black] (0cm,0cm) -- (240:1cm);
            % dot at end point
            \filldraw[black] (240:1cm) circle(0.8pt);

            %g02 phi02
            \draw[black] (0cm,0cm) -- (240:1cm);
            % dot at end point
            \filldraw[black] (240:1cm) circle(0.4pt);
            %g10 phi10
            \draw[black] (0cm,0cm) -- (120:1cm);
            % dot at end point
            \filldraw[black] (120:1cm) circle(0.4pt);

            % text at the end
            \draw (120:1.32cm) node[fill=white] {\begin{tabular}{c} $g_{01} \phi^{01}$\\ $g_{10} \phi^{10}$\end{tabular}};
            \draw (240:1.32cm) node[fill=white] {\begin{tabular}{c} $g_{02} \phi^{02}$\\ $g_{20} \phi^{20}$\end{tabular}};
    
            % draw the horizontal and vertical coordinates
            \draw (1.4cm,0cm)  node[above=1pt] {}
                (0cm,1.4cm)  node[fill=white] {$\Im$};
        \end{tikzpicture}}
        \caption{$S(\boldsymbol{\alpha} = 20)$}\label{fig:case_2_s_alpha_20}
    \end{subfigure}%
    \begin{subfigure}[b]{0.33\textwidth}
        \centering
        \resizebox{\textwidth}{!}{
            \begin{tikzpicture}[scale=2.7,cap=round,>=latex]
                % draw the coordinates
                \draw[dashed] (-1.5cm,0cm) -- (1.5cm,0cm) node[right,fill=white] {};
                \draw[dashed] (0cm,-1.5cm) -- (0cm,1.5cm) node[above,fill=white] {};
        
                % draw the unit circle
                \draw[thick] (0cm,0cm) circle(1cm);
    
                %g02 phi02
                \draw[black] (0cm,0cm) -- (120:1cm);
                % dot at end point
                \filldraw[black] (120:1cm) circle(0.8pt);
                %g20 phi20
                \draw[black] (0cm,0cm) -- (240:1cm);
                % dot at end point
                \filldraw[black] (240:1cm) circle(0.8pt);                
    
                %g01 phi01
                \draw[black] (0cm,0cm) -- (240:1cm);
                % dot at end point
                \filldraw[black] (240:1cm) circle(0.4pt);
                %g10 phi10
                \draw[black] (0cm,0cm) -- (120:1cm);
                % dot at end point
                \filldraw[black] (120:1cm) circle(0.4pt);

                % text at the end
                \draw (120:1.32cm) node[fill=white] {\begin{tabular}{c} $g_{02} \phi^{02}$\\ $g_{10} \phi^{10}$\end{tabular}};
                \draw (240:1.32cm) node[fill=white] {\begin{tabular}{c} $g_{01} \phi^{01}$\\ $g_{20} \phi^{20}$\end{tabular}};
        
                % draw the horizontal and vertical coordinates
                \draw (1.4cm,0cm)  node[above=1pt] {}
                    (0cm,1.4cm)  node[fill=white] {$\Im$};
            \end{tikzpicture}}
            \caption{$S(\boldsymbol{\alpha} = 21)$}\label{fig:case_2_s_alpha_21}
    \end{subfigure}%
    \begin{subfigure}[b]{0.33\textwidth}
        \centering
        \resizebox{\textwidth}{!}{
            \begin{tikzpicture}[scale=2.7,cap=round,>=latex]
                % draw the coordinates
                \draw[dashed] (-1.5cm,0cm) -- (1.5cm,0cm) node[right,fill=white] {};
                \draw[dashed] (0cm,-1.5cm) -- (0cm,1.5cm) node[above,fill=white] {};
        
                % draw the unit circle
                \draw[thick] (0cm,0cm) circle(1cm);
    
                %g01 phi01
                \draw[black] (0cm,0cm) -- (0:1cm);
                % dot at end point
                \filldraw[black] (0:1cm) circle(0.8pt);
                
                %g20 phi20
                \draw[black] (0cm,0cm) -- (240:1cm);
                % dot at end point
                \filldraw[black] (240:1cm) circle(0.6pt);               
    
                %g02 phi02
                \draw[black] (0cm,0cm) -- (240:1cm);
                % dot at end point
                \filldraw[black] (240:1cm) circle(0.4pt);

                %g10 phi10
                \draw[black] (0cm,0cm) -- (120:1cm);
                % dot at end point
                \filldraw[black] (120:1cm) circle(0.6pt);

                % text at the end
                \draw (0:1.32cm) node[fill=white] {\hspace{-.3cm} \begin{tabular}{c}$g_{01} \phi^{01}$\\ $g_{02} \phi^{02}$\end{tabular}};
                \draw (120:1.25cm) node[fill=white] {\begin{tabular}{c} $g_{10} \phi^{10}$\end{tabular}};
                \draw (240:1.25cm) node[fill=white] {\begin{tabular}{c} $g_{20} \phi^{20}$\end{tabular}};
        
                % draw the horizontal and vertical coordinates
                \draw (1.4cm,0cm)  node[above=1pt] {}
                    (0cm,1.4cm)  node[fill=white] {$\Im$};
            \end{tikzpicture}}
            \caption{$S(\boldsymbol{\alpha} = 22)$}\label{fig:case_2_s_alpha_22}
    \end{subfigure}
    \caption{Graphical representation of $S({\boldsymbol{\alpha}})$ of a 3-state,2-spin system with two first-order interactions. The parameters are chosen such that spin one prefers to be in state $\alpha = 1$ and spin two prefers to be in state $\alpha = 2$.}
    \label{fig:s_case_2}
\end{figure}

\begin{table}[h]
    \centering
    \caption{Numerical values for S($\boldsymbol{\alpha}$) and P($\boldsymbol{\alpha}$) for a spin model with two first-order interactions.}
    \label{tab:case_2_num_values}
    \begin{subtable}{.3\textwidth}
        \centering
        \caption{r = 0.5}
        \begin{tabular}{ccc}
            \toprule
             $\boldsymbol{\alpha}$ & S($\boldsymbol{\alpha}$) & P($\boldsymbol{\alpha}$)\\
            \midrule
            00 & -1 & 0.024 \\
            01 & -1 & 0.024 \\
            02 & 0.5 & 0.107 \\
            10 & 0.5 & 0.107 \\
            11 & 0.5 & 0.107 \\
            12 & 2 & 0.478 \\
            20 & -1 & 0.024 \\
            21 & -1 & 0.024 \\
            22 & 0.5 & 0.107\\
          \bottomrule
        \end{tabular}
    \end{subtable}%
    \begin{subtable}{.3\textwidth}
        \centering
        \caption{r = 1}
        \begin{tabular}{ccc}
            \toprule
             $\boldsymbol{\alpha}$ & S($\boldsymbol{\alpha}$) & P($\boldsymbol{\alpha}$)\\
            \midrule
            00 & -2 & 0.002 \\
            01 & -2 & 0.002 \\
            02 & 1 & 0.041 \\
            10 & 1 & 0.041 \\
            11 & 1 & 0.041 \\
            12 & 4 & 0.827 \\
            20 & -2 & 0.002 \\
            21 & -2 & 0.002 \\
            22 & 1 & 0.041 \\
          \bottomrule
        \end{tabular}
    \end{subtable}%
    \begin{subtable}{.3\textwidth}
        \centering
        \caption{r = 2}
        \begin{tabular}{ccc}
            \toprule
             $\boldsymbol{\alpha}$ & S($\boldsymbol{\alpha}$) & P($\boldsymbol{\alpha}$)\\
            \midrule
            00 & -4 & 0.000 \\
            01 & -4 & 0.000 \\
            02 & 2 & 0.002 \\
            10 & 2 & 0.002 \\
            11 & 2 & 0.002 \\
            12 & 8 & 0.990 \\
            20 & -4 & 0.000 \\
            21 & -4 & 0.000 \\
            22 & 2 & 0.002 \\
          \bottomrule
        \end{tabular}
    \end{subtable}
\end{table}

\noindent
\textit{Case 3: One first-order interaction and one second-order interaction}

\begin{itemize}
    \item $g_{01}$ = $r\left[\cos\left( \frac{2\pi}{3}\right) + i \sin\left( \frac{2\pi}{3}\right)\right]$
    \item $g_{02}$ = $r\left[\cos\left( \frac{2\pi}{3}\right) - i \sin\left( \frac{2\pi}{3}\right)\right]$
    \item $g_{12}$ = $r$
    \item $g_{21}$ = $r$
\end{itemize}

\begin{figure}[h!]
    \centering
    \begin{subfigure}[b]{0.33\textwidth}
        \centering
        \resizebox{\textwidth}{!}{
        \begin{tikzpicture}[scale=2.7,cap=round,>=latex]
            % draw the coordinates
            \draw[dashed] (-1.5cm,0cm) -- (1.5cm,0cm) node[right,fill=white] {};
            \draw[dashed] (0cm,-1.5cm) -- (0cm,1.5cm) node[above,fill=white] {};
    
            % draw the unit circle
            \draw[thick] (0cm,0cm) circle(1cm);

            %g01 phi01
            \draw[black] (0cm,0cm) -- (120:1cm);
            % dot at end point
            \filldraw[black] (120:1cm) circle(0.6pt);

            %g02 phi02
            \draw[black] (0cm,0cm) -- (240:1cm);
            % dot at end point
            \filldraw[black] (240:1cm) circle(0.6pt);

            %g12 phi12
            \draw[black] (0cm,0cm) -- (0:1cm);
            % dot at end point
            \filldraw[black] (0:1cm) circle(0.8pt);

            %g21 phi21
            \draw[black] (0cm,0cm) -- (0:1cm);
            % dot at end point
            \filldraw[black] (0:1cm) circle(0.8pt);

            % text at the end
            \draw (0:1.32cm) node[fill=white] {\hspace{-.3cm} \begin{tabular}{c} $g_{12} \phi^{12}$\\ $g_{21} \phi^{21}$\end{tabular}};
            \draw (120:1.32cm) node[fill=white] {\begin{tabular}{c} $g_{01} \phi^{01}$\end{tabular}};
            \draw (240:1.32cm) node[fill=white] {\begin{tabular}{c} $g_{02} \phi^{02}$\end{tabular}};
    
            % draw the horizontal and vertical coordinates
            \draw (1.4cm,0cm)  node[above=1pt] {}
                (0cm,1.4cm)  node[fill=white] {$\Im$};
        \end{tikzpicture}}
        \caption{$S(\boldsymbol{\alpha} = 00)$}\label{fig:case_3_s_alpha_00}
    \end{subfigure}%
    \begin{subfigure}[b]{0.33\textwidth}
        \centering
        \resizebox{\textwidth}{!}{
            \begin{tikzpicture}[scale=2.7,cap=round,>=latex]
                % draw the coordinates
                \draw[dashed] (-1.5cm,0cm) -- (1.5cm,0cm) node[right,fill=white] {};
                \draw[dashed] (0cm,-1.5cm) -- (0cm,1.5cm) node[above,fill=white] {};
        
                % draw the unit circle
                \draw[thick] (0cm,0cm) circle(1cm);
    
                %g02 phi02
                \draw[black] (0cm,0cm) -- (120:1cm);
                % dot at end point
                \filldraw[black] (120:1cm) circle(0.8pt);             
    
                %g01 phi01
                \draw[black] (0cm,0cm) -- (240:1cm);
                % dot at end point
                \filldraw[black] (240:1cm) circle(0.8pt);

                %g12 phi12
                \draw[black] (0cm,0cm) -- (240:1cm);
                % dot at end point
                \filldraw[black] (240:1cm) circle(0.8pt);

                %g21 phi21
                \draw[black] (0cm,0cm) -- (120:1cm);
                % dot at end point
                \filldraw[black] (120:1cm) circle(0.8pt);

                % text at the end
                \draw (120:1.32cm) node[fill=white] {\begin{tabular}{c} $g_{02} \phi^{02}$\\ $g_{21} \phi^{21}$\end{tabular}};
                \draw (240:1.32cm) node[fill=white] {\begin{tabular}{c} $g_{01} \phi^{01}$\\ $g_{12} \phi^{12}$\end{tabular}};
        
                % draw the horizontal and vertical coordinates
                \draw (1.4cm,0cm)  node[above=1pt] {}
                    (0cm,1.4cm)  node[fill=white] {$\Im$};
            \end{tikzpicture}}
            \caption{$S(\boldsymbol{\alpha} = 01)$}\label{fig:case_3_s_alpha_01}
    \end{subfigure}%
    \begin{subfigure}[b]{0.33\textwidth}
        \centering
        \resizebox{\textwidth}{!}{
            \begin{tikzpicture}[scale=2.7,cap=round,>=latex]
                % draw the coordinates
                \draw[dashed] (-1.5cm,0cm) -- (1.5cm,0cm) node[right,fill=white] {};
                \draw[dashed] (0cm,-1.5cm) -- (0cm,1.5cm) node[above,fill=white] {};
        
                % draw the unit circle
                \draw[thick] (0cm,0cm) circle(1cm);
    
                %g01 phi01
                \draw[black] (0cm,0cm) -- (0:1cm);
                % dot at end point
                \filldraw[black] (0:1cm) circle(0.8pt);           
    
                %g02 phi02
                \draw[black] (0cm,0cm) -- (240:1cm);
                % dot at end point
                \filldraw[black] (240:1cm) circle(0.4pt);

                %g12 phi12
                \draw[black] (0cm,0cm) -- (120:1cm);
                % dot at end point
                \filldraw[black] (120:1cm) circle(0.6pt);

                %g21 phi21
                \draw[black] (0cm,0cm) -- (240:1cm);
                % dot at end point
                \filldraw[black] (240:1cm) circle(0.6pt);

                % text at the end
                \draw (0:1.32cm) node[fill=white] {\hspace{-.3cm} \begin{tabular}{c}$g_{01} \phi^{01}$\\ $g_{02} \phi^{02}$\end{tabular}};
                \draw (120:1.25cm) node[fill=white] {\begin{tabular}{c} $g_{12} \phi^{12}$\end{tabular}};
                \draw (240:1.25cm) node[fill=white] {\begin{tabular}{c} $g_{21} \phi^{21}$\end{tabular}};
        
                % draw the horizontal and vertical coordinates
                \draw (1.4cm,0cm)  node[above=1pt] {}
                    (0cm,1.4cm)  node[fill=white] {$\Im$};
            \end{tikzpicture}}
            \caption{$S(\boldsymbol{\alpha} = 02)$}\label{fig:case_3_s_alpha_02}
    \end{subfigure}
    \begin{subfigure}[b]{0.33\textwidth}
        \centering
        \resizebox{\textwidth}{!}{
        \begin{tikzpicture}[scale=2.7,cap=round,>=latex]
            % draw the coordinates
            \draw[dashed] (-1.5cm,0cm) -- (1.5cm,0cm) node[right,fill=white] {};
            \draw[dashed] (0cm,-1.5cm) -- (0cm,1.5cm) node[above,fill=white] {};
    
            % draw the unit circle
            \draw[thick] (0cm,0cm) circle(1cm);

            %g01 phi01
            \draw[black] (0cm,0cm) -- (120:1cm);
            % dot at end point
            \filldraw[black] (120:1cm) circle(0.6pt);

            %g02 phi02
            \draw[black] (0cm,0cm) -- (240:1cm);
            % dot at end point
            \filldraw[black] (240:1cm) circle(0.6pt);

            %g12 phi12
            \draw[black] (0cm,0cm) -- (120:1cm);
            % dot at end point
            \filldraw[black] (120:1cm) circle(0.8pt);

            %g21 phi21
            \draw[black] (0cm,0cm) -- (240:1cm);
            % dot at end point
            \filldraw[black] (240:1cm) circle(0.8pt);

            % text at the end
            \draw (120:1.32cm) node[fill=white] {\begin{tabular}{c} $g_{01} \phi^{01}$\\$g_{12} \phi^{12}$\end{tabular}};
            \draw (240:1.32cm) node[fill=white] {\begin{tabular}{c} $g_{02} \phi^{02}$\\$g_{21} \phi^{21}$\end{tabular}};
    
            % draw the horizontal and vertical coordinates
            \draw (1.4cm,0cm)  node[above=1pt] {}
                (0cm,1.4cm)  node[fill=white] {$\Im$};
        \end{tikzpicture}}
        \caption{$S(\boldsymbol{\alpha} = 10)$}\label{fig:case_3_s_alpha_10}
    \end{subfigure}%
    \begin{subfigure}[b]{0.33\textwidth}
        \centering
        \resizebox{\textwidth}{!}{
            \begin{tikzpicture}[scale=2.7,cap=round,>=latex]
                % draw the coordinates
                \draw[dashed] (-1.5cm,0cm) -- (1.5cm,0cm) node[right,fill=white] {};
                \draw[dashed] (0cm,-1.5cm) -- (0cm,1.5cm) node[above,fill=white] {};
        
                % draw the unit circle
                \draw[thick] (0cm,0cm) circle(1cm);
    
                %g02 phi02
                \draw[black] (0cm,0cm) -- (120:1cm);
                % dot at end point
                \filldraw[black] (120:1cm) circle(0.6pt);             
    
                %g01 phi01
                \draw[black] (0cm,0cm) -- (240:1cm);
                % dot at end point
                \filldraw[black] (240:1cm) circle(0.6pt);

                %g12 phi12
                \draw[black] (0cm,0cm) -- (0:1cm);
                % dot at end point
                \filldraw[black] (0:1cm) circle(0.8pt);

                %g21 phi21
                \draw[black] (0cm,0cm) -- (0:1cm);
                % dot at end point
                \filldraw[black] (0:1cm) circle(0.8pt);

                % text at the end
                \draw (0:1.32cm) node[fill=white] {\hspace{-.3cm} \begin{tabular}{c} $g_{12} \phi^{12}$\\ $g_{21} \phi^{21}$\end{tabular}};
                \draw (120:1.32cm) node[fill=white] {\begin{tabular}{c} $g_{02} \phi^{02}$\end{tabular}};
                \draw (240:1.32cm) node[fill=white] {\begin{tabular}{c} $g_{01} \phi^{01}$\end{tabular}};
        
                % draw the horizontal and vertical coordinates
                \draw (1.4cm,0cm)  node[above=1pt] {}
                    (0cm,1.4cm)  node[fill=white] {$\Im$};
            \end{tikzpicture}}
            \caption{$S(\boldsymbol{\alpha} = 11)$}\label{fig:case_3_s_alpha_11}
    \end{subfigure}%
    \begin{subfigure}[b]{0.33\textwidth}
        \centering
        \resizebox{\textwidth}{!}{
            \begin{tikzpicture}[scale=2.7,cap=round,>=latex]
                % draw the coordinates
                \draw[dashed] (-1.5cm,0cm) -- (1.5cm,0cm) node[right,fill=white] {};
                \draw[dashed] (0cm,-1.5cm) -- (0cm,1.5cm) node[above,fill=white] {};
        
                % draw the unit circle
                \draw[thick] (0cm,0cm) circle(1cm);
    
                %g01 phi01
                \draw[black] (0cm,0cm) -- (0:1cm);
                % dot at end point
                \filldraw[black] (0:1cm) circle(0.8pt);             
    
                %g02 phi02
                \draw[black] (0cm,0cm) -- (0:1cm);
                % dot at end point
                \filldraw[black] (0:1cm) circle(0.8pt);

                %g12 phi12
                \draw[black] (0cm,0cm) -- (240:1cm);
                % dot at end point
                \filldraw[black] (240:1cm) circle(0.6pt);

                %g21 phi21
                \draw[black] (0cm,0cm) -- (120:1cm);
                % dot at end point
                \filldraw[black] (120:1cm) circle(0.6pt);

                % text at the end
                \draw (0:1.32cm) node[fill=white] {\hspace{-.3cm} \begin{tabular}{c}$g_{01} \phi^{01}$\\ $g_{02} \phi^{02}$\end{tabular}};
                \draw (120:1.32cm) node[fill=white] {\begin{tabular}{c} $g_{21} \phi^{21}$\end{tabular}};
                \draw (240:1.32cm) node[fill=white] {\begin{tabular}{c} $g_{12} \phi^{12}$\end{tabular}};
        
                % draw the horizontal and vertical coordinates
                \draw (1.4cm,0cm)  node[above=1pt] {}
                    (0cm,1.4cm)  node[fill=white] {$\Im$};
            \end{tikzpicture}}
            \caption{$S(\boldsymbol{\alpha} = 12)$}\label{fig:case_3_s_alpha_12}
    \end{subfigure}
    \begin{subfigure}[b]{0.33\textwidth}
        \centering
        \resizebox{\textwidth}{!}{
        \begin{tikzpicture}[scale=2.7,cap=round,>=latex]
            % draw the coordinates
            \draw[dashed] (-1.5cm,0cm) -- (1.5cm,0cm) node[right,fill=white] {};
            \draw[dashed] (0cm,-1.5cm) -- (0cm,1.5cm) node[above,fill=white] {};
    
            % draw the unit circle
            \draw[thick] (0cm,0cm) circle(1cm);

            %g01 phi01
            \draw[black] (0cm,0cm) -- (120:1cm);
            % dot at end point
            \filldraw[black] (120:1cm) circle(0.8pt);

            %g02 phi02
            \draw[black] (0cm,0cm) -- (240:1cm);
            % dot at end point
            \filldraw[black] (240:1cm) circle(0.8pt);

            %g12 phi12
            \draw[black] (0cm,0cm) -- (240:1cm);
            % dot at end point
            \filldraw[black] (240:1cm) circle(0.8pt);

            %g21 phi21
            \draw[black] (0cm,0cm) -- (120:1cm);
            % dot at end point
            \filldraw[black] (120:1cm) circle(0.8pt);

            % text at the end
            \draw (120:1.32cm) node[fill=white] {\begin{tabular}{c} $g_{01} \phi^{01}$\\ $g_{21} \phi^{21}$\end{tabular}};
            \draw (240:1.32cm) node[fill=white] {\begin{tabular}{c} $g_{02} \phi^{02}$\\ $g_{12} \phi^{12}$\end{tabular}};
    
            % draw the horizontal and vertical coordinates
            \draw (1.4cm,0cm)  node[above=1pt] {}
                (0cm,1.4cm)  node[fill=white] {$\Im$};
        \end{tikzpicture}}
        \caption{$S(\boldsymbol{\alpha} = 20)$}\label{fig:case_3_s_alpha_20}
    \end{subfigure}%
    \begin{subfigure}[b]{0.33\textwidth}
        \centering
        \resizebox{\textwidth}{!}{
            \begin{tikzpicture}[scale=2.7,cap=round,>=latex]
                % draw the coordinates
                \draw[dashed] (-1.5cm,0cm) -- (1.5cm,0cm) node[right,fill=white] {};
                \draw[dashed] (0cm,-1.5cm) -- (0cm,1.5cm) node[above,fill=white] {};
        
                % draw the unit circle
                \draw[thick] (0cm,0cm) circle(1cm);
    
                %g02 phi02
                \draw[black] (0cm,0cm) -- (120:1cm);
                % dot at end point
                \filldraw[black] (120:1cm) circle(0.8pt);              
    
                %g01 phi01
                \draw[black] (0cm,0cm) -- (240:1cm);
                % dot at end point
                \filldraw[black] (240:1cm) circle(0.8pt);
        
                %g12 phi12
                \draw[black] (0cm,0cm) -- (120:1cm);
                % dot at end point
                \filldraw[black] (120:1cm) circle(0.8pt);

                %g21 phi21
                \draw[black] (0cm,0cm) -- (240:1cm);
                % dot at end point
                \filldraw[black] (240:1cm) circle(0.8pt);

                % text at the end
                \draw (120:1.32cm) node[fill=white] {\begin{tabular}{c} $g_{02} \phi^{02}$\\ $g_{12} \phi^{12}$\end{tabular}};
                \draw (240:1.32cm) node[fill=white] {\begin{tabular}{c} $g_{01} \phi^{01}$\\ $g_{21} \phi^{21}$\end{tabular}};
        
                % draw the horizontal and vertical coordinates
                \draw (1.4cm,0cm)  node[above=1pt] {}
                    (0cm,1.4cm)  node[fill=white] {$\Im$};
            \end{tikzpicture}}
            \caption{$S(\boldsymbol{\alpha} = 21)$}\label{fig:case_3_s_alpha_21}
    \end{subfigure}%
    \begin{subfigure}[b]{0.33\textwidth}
        \centering
        \resizebox{\textwidth}{!}{
            \begin{tikzpicture}[scale=2.7,cap=round,>=latex]
                % draw the coordinates
                \draw[dashed] (-1.5cm,0cm) -- (1.5cm,0cm) node[right,fill=white] {};
                \draw[dashed] (0cm,-1.5cm) -- (0cm,1.5cm) node[above,fill=white] {};
        
                % draw the unit circle
                \draw[thick] (0cm,0cm) circle(1cm);
    
                %g01 phi01
                \draw[black] (0cm,0cm) -- (0:1cm);
                % dot at end point
                \filldraw[black] (0:1cm) circle(0.8pt);              
    
                %g02 phi02
                \draw[black] (0cm,0cm) -- (0:1cm);
                % dot at end point
                \filldraw[black] (0:1cm) circle(0.4pt);

                %g12 phi12
                \draw[black] (0cm,0cm) -- (0:1cm);
                % dot at end point
                \filldraw[black] (0:1cm) circle(1pt);
    
                %g21 phi21
                \draw[black] (0cm,0cm) -- (0:1cm);
                % dot at end point
                \filldraw[black] (0:1cm) circle(0.8pt);

                % text at the end
                \draw (0:1.32cm) node[fill=white] {\hspace{-.3cm} \begin{tabular}{c}$g_{01} \phi^{01}$\\ $g_{02} \phi^{02}$\\ $g_{12} \phi^{12}$\\ $g_{21} \phi^{21}$\end{tabular}};
        
                % draw the horizontal and vertical coordinates
                \draw (1.4cm,0cm)  node[above=1pt] {}
                    (0cm,1.4cm)  node[fill=white] {$\Im$};
            \end{tikzpicture}}
            \caption{$S(\boldsymbol{\alpha} = 22)$}\label{fig:case_3_s_alpha_22}
    \end{subfigure}
    \caption{Graphical representation of $S({\boldsymbol{\alpha}})$ of a 3-state,2-spin system with a first-order interaction, $\phi^{01}$ and $\phi^{01}$, and a second order interaction, $\phi^{12}$ and $\phi^{21}$. The parameters for the first-order interaction are chosen such that states where $\alpha_2 = 2$ are preferred and the parameters of the second-order interactions such that states for which  $\sum_{j \in \mu} \alpha_j \mu_j =  0\mod3$ are preferred.}
    \label{fig:s_case_3}
\end{figure}

\begin{table}[h]
    \centering
    \caption{Numerical values for S($\boldsymbol{\alpha}$) and P($\boldsymbol{\alpha}$) for a spin model with a first-order interaction, $\phi^{01}$ and $\phi^{01}$, and a second order interaction, $\phi^{12}$ and $\phi^{21}$.}
    \label{tab:case_2_num_values}
    \begin{subtable}{.3\textwidth}
        \centering
        \caption{r = 0.5}
        \begin{tabular}{ccc}
            \toprule
             $\boldsymbol{\alpha}$ & S($\boldsymbol{\alpha}$) & P($\boldsymbol{\alpha}$)\\
            \midrule
            00 & 0.5 & 0.107 \\
            01 & -1 & 0.024 \\
            02 & 0.5 & 0.107 \\
            10 & -1 & 0.024 \\
            11 & 0.5 & 0.107 \\
            12 & 0.5 & 0.107 \\
            20 & -1 & 0.024 \\
            21 & -1 & 0.024 \\
            22 & 2 & 0.478\\
          \bottomrule
        \end{tabular}
    \end{subtable}%
    \begin{subtable}{.3\textwidth}
        \centering
        \caption{r = 1}
        \begin{tabular}{ccc}
            \toprule
             $\boldsymbol{\alpha}$ & S($\boldsymbol{\alpha}$) & P($\boldsymbol{\alpha}$)\\
            \midrule
            00 & 1 & 0.041 \\
            01 & -2 & 0.002 \\
            02 & 1 & 0.041 \\
            10 & -2 & 0.002 \\
            11 & 1 & 0.041 \\
            12 & 1 & 0.041 \\
            20 & -2 & 0.002 \\
            21 & -2 & 0.002 \\
            22 & 4 & 0.827 \\
          \bottomrule
        \end{tabular}
    \end{subtable}%
    \begin{subtable}{.3\textwidth}
        \centering
        \caption{r = 2}
        \begin{tabular}{ccc}
            \toprule
             $\boldsymbol{\alpha}$ & S($\boldsymbol{\alpha}$) & P($\boldsymbol{\alpha}$)\\
            \midrule
            00 & 2 & 0.002 \\
            01 & -4 & 0.000 \\
            02 & 2 & 0.002 \\
            10 & -4 & 0.000 \\
            11 & 2 & 0.002 \\
            12 & 2 & 0.002 \\
            20 & -4 & 0.000 \\
            21 & -4 & 0.000 \\
            22 & 8 & 0.990 \\
          \bottomrule
        \end{tabular}
    \end{subtable}
\end{table}
