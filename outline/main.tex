%-------------------------------------------------------------------------------
% LATEX TEMPLATE ARTIKEL

%	PACKAGES EN DOCUMENT CONFIGURATIE
%-------------------------------------------------------------------------------

\documentclass{uva-inf-article}
\usepackage[english]{babel}
\usepackage{graphicx}
\usepackage{amsmath}
\usepackage[title]{appendix}

\usepackage{mathtools} % \shortintertext{}
%Theoremstyle etc. 
\usepackage{amsthm}
\addtolength{\jot}{8pt} % Extra spacing for equations
% Define new theorem "Definition" 
\theoremstyle{definition}
\newtheorem{definition}{Definition}[section]
\newtheorem{corollary}{Corollary}
% Multicolumn environment 
\usepackage{multicol}
% position pictures in text
\usepackage{float}
% line break in table cells
\usepackage{makecell}
% Color table cells
\usepackage{xcolor, colortbl}
% number equations within one section 
\numberwithin{equation}{section}
% Format picture captions
\usepackage[margin={-30pt,20pt}, indention= 2em, font=small,labelfont=bf, labelsep=period]{caption}
\usepackage{subcaption}
% Redefine Figure in Multicol 
\newenvironment{Figure}
  {\par\medskip\noindent\minipage{\linewidth}}
  {\endminipage\par\medskip}
%Algorithms
\usepackage{algorithm}
\usepackage[noend]{algpseudocode} % omit if-end, for-end aso

%\usepackage{natbib}
\usepackage[style=ieee]{biblatex}
\addbibresource{references.bib}

%-------------------------------------------------------------------------------
%	GEGEVENS VOOR IN DE TITEL, HEADER EN FOOTER
%-------------------------------------------------------------------------------

% Geef je artikel een logische titel die de inhoud dekt.
\title{Research outline}

% Vul de naam van de opdracht in zoals gegeven door de docent en het type 
% opdracht, bijvoorbeeld 'technisch rapport' of 'essay'.
\assignment{}
\assignmenttype{}

% Vul de volledige namen van alle auteurs in en de corresponderende UvAnetID's.
\authors{Aaron De Clercq\\aaron.de.clercq@student.uva.nl}
\uvanetids{14483610}



% Vul de naam van je tutor, begeleider (mentor), of docent / vakcoördinator in.
% Vermeld in ieder geval de naam van diegene die het artikel nakijkt!
\tutor{Ebo Peerbooms}
\mentor{}
\docent{Clélia de Mulatier}

% Vul hier de naam van je tutorgroep, werkgroep, of practicumgroep in.
\group{}

% Vul de naam van de cursus in en de cursuscode, te vinden op o.a. DataNose.
\course{}
\courseid{}

% Dit is de datum die op het document komt te staan. Standaard is dat vandaag.
\date{\today}

%-------------------------------------------------------------------------------
%	VOORPAGINA 
%-------------------------------------------------------------------------------

\begin{document}
\maketitle
%-------------------------------------------------------------------------------
%	INHOUDSOPGAVE EN ABSTRACT
%-------------------------------------------------------------------------------
% Niet toevoegen bij een kort artikel, zeg minder dan 10 pagina's!

%TC:ignore
%\tableofcontents
%\begin{abstract}
%\end{abstract}
%TC:endignore

%-------------------------------------------------------------------------------
%	INHOUD
%-------------------------------------------------------------------------------
% Hanteer bij benadering IMRAD: Introduction, Method, Results, Discussion.
\justifying
\begin{multicols}{2}
\section{Introduction}

In many systems higher-order interactions take place that are vital to understand the system\cite{Battiston2021}.
As conducting science is considered to be an iterative process between theory and experiment\cite{Box1976}, one approach or starting point of investigating such systems is to infer a model from observations of the system\cite{Nguyen2017}.
Although the model that fits best given the observed data won't capture the underlying mechanisms of the system perfectly, it does provide more insight into the system and can allow us to make predictions for future observations.
When the microscopic elements of the system can be approximated as spin variables, the Ising model has been used to infer patterns from high-dimensional datasets\cite{Nguyen2017}.
A priori we can't know which interactions are relevant, which means that we would have to consider all possible models with every possible combination of interactions. 
As the number of models scales superexponentially with the number of microscopic variables\cite{Beretta2018}, finding the best fit model is unfeasible for most systems.

\subsubsection*{Restrict the search to MCMs}

\begin{itemize}
    \item Considering all possible models is unfeasible.
    \item Previously, search restricted to only pairwise interactions, assumed to be a good approximation but by definition no higher-order interactions.
    \item Approach to consider only simple models, because they are easily falsifiable and capture only the essence of the data. Contrast with big neural networks in Machine learning (Universal approximators)
    \item Turns out that pairwise models are not simple\cite{Beretta2018} but MCMs are so search restricted to MCMs\cite{Demulatier2020}.
\end{itemize}

\subsubsection*{Why do we want to extend the model selection from binary to discrete data?}

Every microscopic variable in the system can be in one of many states instead of two states which results in a more accurate description for some systems.
It has been mentioned that depending on the nature of the system the restriction of binarizing the data is too restrictive\cite{Nguyen2017}, but some concrete examples should be given.

\subsubsection*{How was previously dealt with discrete data?}

Inference of Potts model. So far I can only find work where search was restricted to pairwise interactions\cite{Barton2016}.

\subsubsection*{Why do we use Bayesian model selection?}

The following references give an overview of different modelling selection methods and criteria such as AIC and BIC:

\begin{itemize}
    \item \citetitle{Zucchini2000}\cite{Zucchini2000}
    \item \citetitle{Ding2018}\cite{Ding2018}
\end{itemize}


%\input{theory.tex}
%\input{methodology.tex}
%\input{results.tex}
%\input{conclusion.tex}
\printbibliography
\end{multicols}
%-------------------------------------------------------------------------------
%	REFERENCES
%-------------------------------------------------------------------------------

%-------------------------------------------------------------------------------
%	BIJLAGEN 
%-------------------------------------------------------------------------------



%-------------------------------------------------------------------------------
\end{document}
