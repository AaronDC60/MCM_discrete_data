%-------------------------------------------------------------------------------
% LATEX TEMPLATE ARTIKEL

%	PACKAGES EN DOCUMENT CONFIGURATIE
%-------------------------------------------------------------------------------

\documentclass{uva-inf-article}
\usepackage[english]{babel}
\usepackage{graphicx}
\usepackage{amsmath}
\usepackage{amssymb}
\usepackage[title]{appendix}

\usepackage{booktabs}
\usepackage{mathtools} % \shortintertext{}
\usepackage{physics}
%Theoremstyle etc. 
\usepackage{amsthm}
\addtolength{\jot}{8pt} % Extra spacing for equations
% Define new theorem "Definition" 
\theoremstyle{definition}
\newtheorem{definition}{Definition}[section]
\newtheorem{corollary}{Corollary}
% Multicolumn environment 
\usepackage{multicol}
% position pictures in text
\usepackage{float}
% line break in table cells
\usepackage{makecell}
% Color table cells
\usepackage{xcolor, colortbl}
% number equations within one section 
\numberwithin{equation}{section}
% Format picture captions
\usepackage[margin={-30pt,20pt}, indention= 2em, font=small,labelfont=bf, labelsep=period]{caption}
\usepackage{subcaption}
% Redefine Figure in Multicol 
\newenvironment{Figure}
  {\par\medskip\noindent\minipage{\linewidth}}
  {\endminipage\par\medskip}
%Algorithms
\usepackage{algorithm}
\usepackage[noend]{algpseudocode} % omit if-end, for-end aso

\usepackage[style=ieee]{biblatex}
\addbibresource{references.bib}

\allowdisplaybreaks

%-------------------------------------------------------------------------------
%	GEGEVENS VOOR IN DE TITEL, HEADER EN FOOTER
%-------------------------------------------------------------------------------

% Geef je artikel een logische titel die de inhoud dekt.
\title{Research  notes}

% Vul de naam van de opdracht in zoals gegeven door de docent en het type 
% opdracht, bijvoorbeeld 'technisch rapport' of 'essay'.
\assignment{}
\assignmenttype{}

% Vul de volledige namen van alle auteurs in en de corresponderende UvAnetID's.
\authors{Aaron De Clercq\\aaron.de.clercq@student.uva.nl}
\uvanetids{14483610}



% Dit is de datum die op het document komt te staan. Standaard is dat vandaag.
\date{\today}

%-------------------------------------------------------------------------------
%	VOORPAGINA 
%-------------------------------------------------------------------------------

\begin{document}
\maketitle
%-------------------------------------------------------------------------------
%	INHOUDSOPGAVE EN ABSTRACT
%-------------------------------------------------------------------------------
% Niet toevoegen bij een kort artikel, zeg minder dan 10 pagina's!

%TC:ignore
%\tableofcontents
%\begin{abstract}
%\end{abstract}
%TC:endignore

%-------------------------------------------------------------------------------
%	INHOUD
%-------------------------------------------------------------------------------
% Hanteer bij benadering IMRAD: Introduction, Method, Results, Discussion.
\justifying

%\input{lit_study}

\section{Theoretical background}

This section covers the theoretical concepts necessary to understand the research project.

\subsection{Spin models}

One approach to represent a system is through a spin model. Assume a given system with $n$ components where a single observation of the system yields one value for every component.
Such an observation can be seen as a specific state of the system.

\subsubsection{Binary spin operators}

In case that the observations can be binarized, the state of the system can be written as a spin configuration,

\begin{equation}
    \mathbf{s} = (s_1 \dots s_n),
\end{equation}

\noindent
of $n$ spin variables, where every variable takes a value of either +1 or -1.

\noindent
In a spin configuration, we can express an interaction between a subset $\mu$ of these spins using a spin operator which is defined as the product of all spins in the subset,

\begin{equation}\label{eq:spin_op}
    \phi^\mu(\mathbf{s}) = \prod_{i \in \mu} s_i.
\end{equation}

\noindent
Given that all spin values are either +1 or -1, the spin operator will be +1 if an even number of spins are -1 and it will be -1 if an odd number of spins are -1.
The sum over all possible spin configurations of a spin operator is equal to zero,

\begin{equation}\label{eq:sum_over_conf}
    \sum_{\mathbf{s} \in \mathbf{S}} \phi^\mu(\mathbf{s}) = 0,
\end{equation}

\noindent
because half of the spin configurations can be written as a spin configuration in the other half with exactly one spinflip.
In a system with $n$ spin variables, there are $2^n - 1$ possible subsets, which means that there are $2^n - 1$ different spin operators.

\begin{definition}
    A set of operators, $\Omega$ is called orthogonal if

    \begin{equation}\label{eq:ortho}
        \frac{1}{2^n} \sum_{\mathbf{s} \in \mathbf{S}} \phi^\mu(\mathbf{s}) \phi^\nu(\mathbf{s}) = \delta_{\mu\nu} \qquad \forall \;\; \phi^\mu, \phi^\nu \in \Omega
    \end{equation}
\end{definition}

\begin{definition}
    A set of operators, $\Omega$ is called complete if

    \begin{equation}\label{eq:complete}
        \frac{1}{2^n} \sum_{\mu \in \Omega} \phi^\mu(\mathbf{s}) \phi^\mu(\mathbf{s}^\prime) = \delta_{\mathbf{s}\mathbf{s}^\prime},
    \end{equation}
    as this yields a unique relation between the spin configuration and the values for every operator in the set.
\end{definition}

\noindent
If we define the identity operator, $\phi^0 = 1$, then the set of operators $\Omega_n = \{\phi^\mu(\mathbf{s})\}_{\mu \in \{0 \dots 2^n-1\}}$ is a complete and orthogonal set.
For orthogonality, we can rewrite Equation \ref{eq:ortho}, use Equation \ref{eq:sum_over_conf} if $\mu$ and $\nu$ are different and use the definition of the identity operator if $\mu$ and $\nu$ are the same.

\begin{align*}
    \frac{1}{2^n} \sum_{\mathbf{s} \in \mathbf{S}} \phi^\mu(\mathbf{s}) \phi^\nu(\mathbf{s}) &= \frac{1}{2^n} \sum_{\mathbf{s} \in \mathbf{S}} \phi^{\mu \oplus \nu (\mathbf{s})}\\
    &= \delta_{\mu\nu}
\end{align*}

\noindent
To show the completeness of $\Omega_n$, we can plug Equation \ref{eq:spin_op} into Equation \ref{eq:complete} and rewrite the sum over all spin operators as a product of sums using the bitstring representation of the spin operators.

\begin{align*}
    \frac{1}{2^n} \sum_{\mu \in \Omega_n} \phi^\mu(\mathbf{s}) \phi^\mu(\mathbf{s}^\prime) &= \frac{1}{2^n} \sum_{\mu=0}^{2^n-1} \prod_{i \in \mu} s_i s_i^\prime \\
    &= \frac{1}{2^n} \sum_{\alpha_1 = 0,1} \dots \sum_{\alpha_n = 0,1} \prod_{i = 1}^{n} (s_i s_i^\prime)^{\alpha_i} \\
    &= \delta_{\mathbf{s}\mathbf{s}^\prime}
\end{align*}

\noindent
If the two spin configurations are the same, $s_i s_i^\prime$ will always be equal to 1 while it will be half of the time +1 and half of the time -1 for two configurations that are different.

\begin{definition}
    A set of $n$ spin operators that can fully generate $\Omega_n$ by taking linear combinations of the individual spin operators is called a generating set of $\Omega_n$.
\end{definition}

\begin{definition}
    A set of spin operators that for which the product of every subset does not yield the identity operator is called a set of independent operators.
\end{definition}

\subsubsection{Discrete spin operators}

In case a system of $n$ variables where each variable can have $q$ different values, labeled from $0$ to $q-1$, a state of the system is a vector $\boldsymbol{\alpha}= (\alpha_1 \dots \alpha_n) \in {(\mathbb{Z}/q\mathbb{Z})}^n$.
We can still see these states as spins by representing them as vectors on a unit circle with an angle of $\frac{2\pi}{q}\alpha$.
The mapping between the value $\alpha_j$ of a variable and the spin value $s_j$ is then given by the following relation

\begin{equation}
    s_j = e^{\frac{2\pi i \alpha_j}{q}}.
\end{equation}

\begin{figure}[h]
    \centering
    \begin{tikzpicture}[scale=4,cap=round,>=latex]
        % draw the coordinates
        \draw[dashed] (-1.5cm,0cm) -- (1.5cm,0cm) node[right,fill=white] {};
        \draw[dashed] (0cm,-1.5cm) -- (0cm,1.5cm) node[above,fill=white] {};

        % draw the unit circle
        \draw[thick] (0cm,0cm) circle(1cm);

        \foreach \x in {0,1,2} {
            % lines from center to point
            \draw[gray] (0cm,0cm) -- (\x*120:1cm);
            % dots at each point
            \filldraw[black] (\x*120:1cm) circle(0.4pt);
            % draw each angle in degrees
            \draw (\x*120:0.6cm) node[fill=white] {$\alpha = \x$};
    }
    \foreach \x/\xtext/\y in {
        % alpha = 1
        120/\text{cos}\left(\frac{2\pi}{3}\right)/\text{sin}\left(\frac{2\pi}{3}\right),
        % alpha = 2
        240/\text{cos}\left(\frac{4\pi}{3}\right)/\text{sin}\left(\frac{4\pi}{3}\right)}
            \draw (\x:1.25cm) node[fill=white] {$\left(\xtext, \y\right)$};

        % draw the horizontal and vertical coordinates
        \draw (-1.25cm,0cm) node[above=1pt] {$(-1,0)$}
            (1.25cm,0cm)  node[above=1pt] {$(1,0)$}
            (0cm,-1.25cm) node[fill=white] {$(0,-i)$}
            (0cm,1.25cm)  node[fill=white] {$(0,i)$};
    \end{tikzpicture}
    \caption{All states in a 3-state system drawn as the cube roots of unity.}
    \label{fig:roots_of_unity}
\end{figure}

\noindent
These spin values are equivalent to the q-th roots of unity, which are drawn in Figure \ref{fig:roots_of_unity} for q equal to three.
Analogous to the binary case, we can define a spin operator,

\begin{equation}
    \phi^{\boldsymbol{\mu}}(\mathbf{s}) = \prod_{j=1}^{n} s_j^{\mu_j},
\end{equation}

\noindent
with $\boldsymbol{\mu} \in {(\mathbb{Z}/q\mathbb{Z})}^n$. With the bijective relation between $\mathbf{s}$ and $\boldsymbol{\alpha}$ we can also write the spin operators in terms of $\boldsymbol{\alpha}$,

\begin{equation}
    \phi^{\boldsymbol{\mu}}(\boldsymbol{\alpha}) = e^{\frac{2\pi i}{q} \sum_{j=1}^{n}  \alpha_j \mu_j},
\end{equation}

\noindent
In total there are $q^n$ spin operators that form a finite multiplicative group that is equal to the set of the q-th roots of unity.
Applying a spin operator $\phi^{\boldsymbol{\mu}}$ on a state $\boldsymbol{\alpha}$ can be seen as a rotation around the unit circle starting from the zero state over an angle $\sum_{j=1}^{n}  \alpha_j \mu_j$.
Similar as in the binary case, we can construct a $q$ x $q$ spin operator matrix.

\begin{equation}
    \mathbf{S}^{(q)} = \begin{bmatrix}
        \phi^0(\alpha=0) & \phi^1(\alpha=0) & \hdots &  \phi^{q-1}(\alpha=0)\\
        \phi^0(\alpha=1) & \phi^1(\alpha=1) & \hdots &  \phi^{q-1}(\alpha=1)\\
        \vdots & \vdots & \ddots & \vdots \\
        \phi^0(\alpha=q-1) & \phi^1(\alpha=q-1) & \hdots & \phi^{q-1}(\alpha=q-1)\\
    \end{bmatrix}\label{eq:spin_op_matrix}
\end{equation}

\subsubsection{Probability distribution}

\noindent
It was previously mentioned that a spin model can be used to represent system. The idea is that we assume the system to be in some sort of an equilibrium and the observed spin configurations can be seen as sampled from a given probability distribution.
Usually a probability distribution with an exponential form is chosen,

\begin{equation}\label{eq:prob_distr}
    p(\boldsymbol{\alpha}| \mathbf{g}, \mathcal{M}) = \frac{1}{Z(\mathbf{g}, \mathcal{M})} e^{\sum_{\boldsymbol{\mu} \in \mathcal{M}} g_{\boldsymbol{\mu}} \phi^{\boldsymbol{\mu}}(\boldsymbol{\alpha})},
\end{equation}

\noindent
with

\begin{equation}
    Z(\mathbf{g}, \mathcal{M}) = \sum_{\boldsymbol{\alpha} \in {(\mathbb{Z}/q\mathbb{Z})}^n}e^{\sum_{\boldsymbol{\mu} \in \mathcal{M}} g_{\boldsymbol{\mu}} \phi^{\boldsymbol{\mu}}(\boldsymbol{\alpha})}
\end{equation}

\noindent
because this is the distribution that maximizes the entropy under the constraint that the expected value for every spin operator in the model is equal to their value in the observed data, which is shown in Appendix \ref{sec:max_entropy}.
Note that this distribution is similar to the Boltzmann distribution, which is used in statistical mechanics to describe a system at thermal equilibrium.

The model $\mathcal{M}$ is defined as a subset of $\Omega_n \setminus \{ \phi^0 \}$ and $g_{\boldsymbol{\mu}} \in {\mathbb{C}}$ is a model parameter. 
The magnitude of the parameter indicates the strength of the interaction between the spin variables in $\boldsymbol{\mu}$ and the direction of $g_{\boldsymbol{\mu}}$ tells us for which $\boldsymbol{\alpha}$ the contribution of $\phi^{\boldsymbol{\mu}}(\boldsymbol{\alpha})$ to $p(\boldsymbol{\alpha})$ is maximized.

\noindent
If we set $g_{\boldsymbol{\mu}} = 0 \quad \forall \boldsymbol{\mu} \notin \mathcal{M}$ and $g_0 = -\log Z(\mathbf{g}, \mathcal{M})$, we can write the probability distribution as

\begin{equation}
    p(\boldsymbol{\alpha}) =e^{S(\boldsymbol{\alpha})}\label{eq:p_alpha_s_alpha}
\end{equation}

\noindent
with

\begin{align}
    S(\boldsymbol{\alpha}) &= \sum_{\boldsymbol{\mu} \in {(\mathbb{Z}/q\mathbb{Z})}^n} g_{\boldsymbol{\mu}} \phi^{\boldsymbol{\mu}}(\boldsymbol{\alpha}),\notag \\
    &= \bigotimes_{i = 1}^{n} \mathbf{S}^{(q)} \cdot \mathbf{g}.\label{eq:s_alpha}
\end{align}

\noindent
In Appendix \ref{sec:2spin_2states} a small example is worked out for a 2-state,2-spin system.
Note that in the case with more than two states, $\phi^{\boldsymbol{\mu}}(\boldsymbol{\alpha})$ can be a complex number. However, $S(\boldsymbol{\alpha})$ has to be a real function to get real, positive probabilities.
In order to guarantee $S(\boldsymbol{\alpha})$ being real, the following constraint is set for the model parameters.

\begin{equation}
    g_{-\boldsymbol{\mu}} = g_{\boldsymbol{\mu}}^*.
\end{equation}


\noindent
We can also split the model parameter into a real and imaginary part,

\begin{equation}
    g_{\boldsymbol{\mu}} = a_{\boldsymbol{\mu}} + i b_{\boldsymbol{\mu}},
\end{equation}

\noindent
in which case the constraint becomes

\begin{equation}
    \begin{cases}
        a_{-\boldsymbol{\mu}} = a_{\boldsymbol{\mu}}\\
        b_{-\boldsymbol{\mu}} = -b_{\boldsymbol{\mu}}
    \end{cases}
    .
\end{equation}

\noindent
Using these constraints, we can rewrite the function $S(\boldsymbol{\alpha})$,

\begin{align*}
    S(\boldsymbol{\alpha}) =& \sum_{\boldsymbol{\mu} \in {(\mathbb{Z}/q\mathbb{Z})}^n} g_{\boldsymbol{\mu}} \phi^{\boldsymbol{\mu}}(\boldsymbol{\alpha}),\\
    =& \frac{1}{2} \sum_{\boldsymbol{\mu} \in {(\mathbb{Z}/q\mathbb{Z})}^n} g_{\boldsymbol{\mu}} \phi^{\boldsymbol{\mu}}(\boldsymbol{\alpha}) + g_{-\boldsymbol{\mu}} \phi^{-\boldsymbol{\mu}}(\boldsymbol{\alpha}),\\
    =& \frac{1}{2} \sum_{\boldsymbol{\mu} \in {(\mathbb{Z}/q\mathbb{Z})}^n} (a_{\boldsymbol{\mu}} + i b_{\boldsymbol{\mu}}) \left[ \cos \left( \frac{2\pi}{q} \boldsymbol{\alpha} \cdot \boldsymbol{\mu}\right) + i \sin \left( \frac{2\pi}{q} \boldsymbol{\alpha} \cdot \boldsymbol{\mu} \right)\right] \\
    &+ (a_{\boldsymbol{\mu}} - i b_{\boldsymbol{\mu}}) \left[ \cos \left( -\frac{2\pi}{q} \boldsymbol{\alpha} \cdot \boldsymbol{\mu} \right) + i \sin \left( -\frac{2\pi}{q} \boldsymbol{\alpha} \cdot \boldsymbol{\mu}\right)\right],\\
    =&  \frac{1}{2} \sum_{\boldsymbol{\mu} \in {(\mathbb{Z}/q\mathbb{Z})}^n} a_{\boldsymbol{\mu}} \left[ \cos \left( \frac{2\pi}{q} \boldsymbol{\alpha} \cdot \boldsymbol{\mu}\right) + i \sin \left( \frac{2\pi}{q} \boldsymbol{\alpha} \cdot \boldsymbol{\mu} \right) + \cos \left( \frac{2\pi}{q} \boldsymbol{\alpha} \cdot \boldsymbol{\mu}\right) - i \sin \left( \frac{2\pi}{q} \boldsymbol{\alpha} \cdot \boldsymbol{\mu} \right) \right]\\
    &+ b_{\boldsymbol{\mu}} \left[ i \cos \left( \frac{2\pi}{q} \boldsymbol{\alpha} \cdot \boldsymbol{\mu}\right) - \sin \left( \frac{2\pi}{q} \boldsymbol{\alpha} \cdot \boldsymbol{\mu} \right) - i \cos \left( \frac{2\pi}{q} \boldsymbol{\alpha} \cdot \boldsymbol{\mu}\right) - \sin \left( \frac{2\pi}{q} \boldsymbol{\alpha} \cdot \boldsymbol{\mu} \right) \right], \\
    =& \sum_{\boldsymbol{\mu} \in {(\mathbb{Z}/q\mathbb{Z})}^n} a_{\boldsymbol{\mu}} \cos \left( \frac{2\pi}{q} \boldsymbol{\alpha} \cdot \boldsymbol{\mu}\right) - b_{\boldsymbol{\mu}} \sin \left( \frac{2\pi}{q} \boldsymbol{\alpha} \cdot \boldsymbol{\mu} \right).
\end{align*}

\noindent
Representing complex numbers as vectors, $z = \begin{bmatrix} \Re(z) & \Im(z) \end{bmatrix}^\intercal$, allows us to write

\begin{equation}
    g_{\boldsymbol{\mu}} = \begin{bmatrix} a & b \end{bmatrix}^\intercal,
\end{equation}

\begin{equation}
    \phi^{\boldsymbol{\mu}}(\boldsymbol{\alpha}) = \begin{bmatrix} \cos \left( \frac{2\pi}{q} \boldsymbol{\alpha} \cdot \boldsymbol{\mu}\right) & \sin \left( \frac{2\pi}{q} \boldsymbol{\alpha} \cdot \boldsymbol{\mu}\right) \end{bmatrix}^\intercal.
\end{equation}

\noindent
Now, we can recognize that the contribution of $\phi^{\boldsymbol{\mu}}(\boldsymbol{\alpha})$ to $S(\boldsymbol{\alpha})$ is the inner product of $g_{\boldsymbol{\mu}}$ with $\phi^{\boldsymbol{\mu}}(\boldsymbol{\alpha})$.

\begin{align}
    S(\boldsymbol{\alpha}) =& \sum_{\boldsymbol{\mu} \in {(\mathbb{Z}/q\mathbb{Z})}^n} \begin{bmatrix}
        a & -b
    \end{bmatrix}\begin{bmatrix}
        \cos \left( \frac{2\pi}{q} \boldsymbol{\alpha} \cdot \boldsymbol{\mu}\right) \\ \sin \left( \frac{2\pi}{q} \boldsymbol{\alpha} \cdot \boldsymbol{\mu}\right)
    \end{bmatrix}, \notag \\
    =& \sum_{\boldsymbol{\mu} \in {(\mathbb{Z}/q\mathbb{Z})}^n} \begin{bmatrix}
        a & b
    \end{bmatrix}^*\begin{bmatrix}
        \cos \left( \frac{2\pi}{q} \boldsymbol{\alpha} \cdot \boldsymbol{\mu}\right) \\ \sin \left( \frac{2\pi}{q} \boldsymbol{\alpha} \cdot \boldsymbol{\mu}\right)
    \end{bmatrix}, \notag \\
    =& \sum_{\boldsymbol{\mu} \in {(\mathbb{Z}/q\mathbb{Z})}^n} \langle g_{\boldsymbol{\mu}}, \phi^{\boldsymbol{\mu}}(\boldsymbol{\alpha}) \rangle.
\end{align}

\noindent
As shown in Figure \ref{fig:S_a}, this inner product can geometrically be seen as the projection of $g_{\boldsymbol{\mu}}^*$ onto $\phi^{\boldsymbol{\mu}}(\boldsymbol{\alpha})$.
The value of this inner product is

\begin{equation}
    \langle g_{\boldsymbol{\mu}}, \phi^{\boldsymbol{\mu}}(\boldsymbol{\alpha}) \rangle = |g_{\boldsymbol{\mu}}| \cos \theta,
\end{equation}

\noindent
where $\theta$ is the angle between $g_{\boldsymbol{\mu}}^*$ and $\phi^{\boldsymbol{\mu}}(\boldsymbol{\alpha})$.
This means that for a given magnitude of the model parameter, the contribution of $\phi^{\boldsymbol{\mu}}(\boldsymbol{\alpha})$ to $S(\boldsymbol{\alpha})$ is maximized if $g_{\boldsymbol{\mu}}^*$ and $\phi^{\boldsymbol{\mu}}(\boldsymbol{\alpha})$ are exactly aligned.
Note that due to the constraints, the contribution of $\phi^{\boldsymbol{\mu}}(\boldsymbol{\alpha})$ is the same as the contribution of $\phi^{-\boldsymbol{\mu}}(\boldsymbol{\alpha})$.
Therefore, only one of the contributions will be drawn in future examples.

\begin{figure}[h]
    \centering
    \begin{tikzpicture}[scale=4,cap=round,>=latex]
        % draw the coordinates
        \draw[] (-1.3cm,0cm) -- (1.3cm,0cm) node[right,fill=white] {};
        \draw[] (0cm,-1.3cm) -- (0cm,1.3cm) node[above,fill=white] {};

        % draw the unit circle
        \draw[thick] (0cm,0cm) circle(1cm);

        %coordinates for angle
        \coordinate (O) at (0,0);
        \coordinate (G) at (200:1cm);
        \coordinate (P) at (240:1cm);

        %phi_mu
        \draw[blue, ->, very thick] (0cm,0cm) -- (120:1cm);
        \draw (120:1.2cm) node[fill=white] {$\phi^{\boldsymbol{\mu}}(\boldsymbol{\alpha})$};
    
        %phi_-mu
        \draw[blue, ->, very thick] (0cm,0cm) -- (240:1cm);
        \draw (240:1.2cm) node[fill=white] {$\phi^{-\boldsymbol{\mu}}(\boldsymbol{\alpha})$};

        %g1
        %\draw[red, ->, very thick, dashed] (0cm,0cm) -- (260:1cm);
        %\draw (260:1.2cm) node[fill=white] {$g_\mu$};

        %g_mu*
        \draw[red, ->, very thick] (0cm,0cm) -- (160:1cm);
        \draw (160:1.15cm) node[fill=white] {$g_{\boldsymbol{\mu}}^*$};

        %g_-mu*
        \draw[red, ->, very thick] (0cm,0cm) -- (200:1cm);
        \draw (200:1.15cm) node[fill=white] {$g_{-\boldsymbol{\mu}}^*$};

        %a1
        %\draw[dashed] (100:1cm) -- (260:1cm);
        \draw[dashed] (160:1cm) -- (200:1cm);
        \draw [decorate,decoration={brace,amplitude=5pt,raise=0.1ex}]
        (0,0) -- (-0.939692,0) node[midway,yshift=-1em]{$a$};

        %b1
        %\draw[dashed] (260:1cm) -- (270: .9848cm);
        %\draw [decorate,decoration={brace,amplitude=5pt,raise=0.7ex}]
        %(0,0) -- (0,-.9848) node[midway,xshift=.4cm]{$b$};

        %b1 tilde
        \draw[dashed] (160:1cm) -- (90: .3420cm);
        \draw [decorate,decoration={brace,amplitude=5pt, mirror,raise=0.7ex}]
        (0,0) -- (0,.3420) node[midway,xshift=.6cm]{$-b$};

        %projection
        \draw[dashdotted] (160:1cm) -- (120:.7660cm);
        \filldraw[black] (120:.7660cm) circle(0.6pt);
        %\draw [decorate,decoration={brace,amplitude=5pt, mirror,raise=1ex}]
        %(0,0) -- (120:.7660) node[midway,xshift=1.3cm, yshift=0.6cm]{$|g_{\boldsymbol{\mu}}^*| \cos \theta$};

        \draw[dashdotted] (200:1cm) -- (240:.7660cm);
        \filldraw[black] (240:.7660cm) circle(0.6pt);
        \draw pic[-,"\small $\theta$",draw=black,angle radius=30,angle eccentricity=1.3] {angle=G--O--P};
        \draw [decorate,decoration={brace,amplitude=5pt,raise=0.6ex}]
        (0,0) -- (240:.7660) node[midway,xshift=.85cm, yshift=-.5cm]{$|g_{\boldsymbol{\mu}}^*| \cos \theta$};


        % draw the horizontal and vertical coordinates
        \draw (1.3cm,0cm)  node[above=1pt] {$\Re$}
            (0cm,1.3cm)  node[fill=white] {$\Im$};
    \end{tikzpicture}
    \caption{Contribution of $\phi^{\boldsymbol{\mu}}(\boldsymbol{\alpha})$ to $S(\boldsymbol{\alpha})$ drawn as an inner product.}
    \label{fig:S_a}
\end{figure}

\subsubsection{Parameter inference}

For a complete model, there exists a bijective relation between $p(\boldsymbol{\alpha})$ and the model parameters, which allows us to construct an expression for the model parameters in terms of $p(\boldsymbol{\alpha})$.
Starting from Equation \ref{eq:p_alpha_s_alpha} we can write

\begin{align*}
    \log p(\boldsymbol{\alpha}) &= S(\boldsymbol{\alpha}),\\
    &= \sum_{\boldsymbol{\nu} \in {(\mathbb{Z}/q\mathbb{Z})}^n} g_{\boldsymbol{\nu}} \phi^{\boldsymbol{\nu}}(\boldsymbol{\alpha}),
\end{align*}

\noindent
Multiplying both sides by $\left[\phi^{\boldsymbol{\mu}}(\boldsymbol{\alpha})\right]^*$ and summing over all possible states $\boldsymbol{\alpha}$ gives 

\begin{align*}
    \sum_{\boldsymbol{\alpha} \in {(\mathbb{Z}/q\mathbb{Z})}^n} \left[\phi^{\boldsymbol{\mu}}(\boldsymbol{\alpha})\right]^* \log p(\boldsymbol{\alpha}) &= \sum_{\boldsymbol{\alpha} \in {(\mathbb{Z}/q\mathbb{Z})}^n} \sum_{\boldsymbol{\nu} \in {(\mathbb{Z}/q\mathbb{Z})}^n} g_{\boldsymbol{\nu}} \phi^{\boldsymbol{\nu}}(\boldsymbol{\alpha}) \left[\phi^{\boldsymbol{\mu}}(\boldsymbol{\alpha})\right]^*\\
    &= \sum_{\boldsymbol{\alpha} \in {(\mathbb{Z}/q\mathbb{Z})}^n} g_{\boldsymbol{\mu}} + \sum_{\substack{\boldsymbol{\nu} \in {(\mathbb{Z}/q\mathbb{Z})}^n \\ \boldsymbol{\nu} \neq \boldsymbol{\mu}}} g_{\boldsymbol{\nu}} \sum_{\boldsymbol{\alpha} \in {(\mathbb{Z}/q\mathbb{Z})}^n} \phi^{\boldsymbol{\nu} - \boldsymbol{\mu}}(\boldsymbol{\alpha})
\end{align*}

\noindent
Given that the $\sum_{\boldsymbol{\alpha} \in {(\mathbb{Z}/q\mathbb{Z})}^n} \phi^{\boldsymbol{\mu}} = 0 \quad \forall \boldsymbol{\mu} \neq \mathbf{0}$, we obtain that

\begin{align*}
    g_{\boldsymbol{\mu}} = \frac{1}{q^n}  \sum_{\boldsymbol{\alpha} \in {(\mathbb{Z}/q\mathbb{Z})}^n} \left[\phi^{\boldsymbol{\mu}}(\boldsymbol{\alpha})\right]^* \log p(\boldsymbol{\alpha})
\end{align*}
\appendix

\newpage

\section{Derivations}

\subsection{Maximum entropy models}\label{sec:max_entropy}

The Shannon entropy is defined as 

\begin{equation}
    S = - \sum_{\mathbf{s}} p(\mathbf{s}) \: \log(p(\mathbf{s})).
\end{equation}

\noindent
The first constraint is the normalization of the distribution,

\begin{equation*}
    \sum_{\mathbf{s}} p(\mathbf{s}) = 1
\end{equation*}

\noindent
Next, we have a set of constraints to make sure that the expected value for every spin operator $\phi^\mu$ in the model corresponds to the observed average value of that spin operator,

\begin{equation*}
    \sum_{\mathbf{s}} \phi^\mu p(\mathbf{s}) = \expval{\phi^\mu}_{data}. \label{eq:constraints}
\end{equation*}

\noindent
To find a probability distribution that maximizes the entropy while satisfying the constraints we can construct the following Lagrangian function and set the derivative with respect to $p(\mathbf{s})$ equal to zero.

\begin{equation}
    \mathcal{L}[p(\mathbf{s})] = S[p(\mathbf{s})] + \lambda_0 \left( \sum_{\mathbf{s}} p(\mathbf{s}) - 1 \right) + \sum_{\mu \in \mathcal{M}} \alpha_\mu \left( \sum_{\mathbf{s}} \phi^\mu \: p(\mathbf{s}) - \expval{\phi^\mu}_{data} \right)\\
\end{equation}

\noindent
The derivative of $\mathcal{L}[p(\mathbf{s})]$ with respect to $p(\mathbf{s})$ is

\begin{equation*}
    \frac{\partial \mathcal{L}[\mathbf{p}]}{\partial p_\mathbf{s}} = \frac{\partial S[\mathbf{p}]}{\partial p_\mathbf{s}} + \lambda_0 \left( \frac{\partial \left( \sum_{\mathbf{s}} p(\mathbf{s}) - 1 \right)}{\partial p_\mathbf{s}}\right) + \sum_{\mu \in \mathcal{M}} \alpha_\mu \left( \frac{\partial \left( \sum_{\mathbf{s}} \phi^\mu \: p(\mathbf{s}) - \expval{\phi^\mu}_{data} \right)}{\partial p_\mathbf{s}} \right) \\
\end{equation*}

\noindent
For the first term on the right-hand side we find

\begin{align*}
    \frac{\partial S[\mathbf{p}]}{\partial p_\mathbf{s}} &= - \frac{\partial \sum_{\mathbf{s}} p(\mathbf{s}) \: \log(p(\mathbf{s}))}{\partial p_\mathbf{s}}\\
    &= - \frac{\partial \: p_{\mathbf{s}} \: \log(p_{\mathbf{s}})}{\partial p_\mathbf{s}} \\
    &= - \log(p_{\mathbf{s}}) \frac{\partial p_\mathbf{s}}{\partial p_\mathbf{s}} - p_\mathbf{s} \frac{\partial \log(p_\mathbf{s})}{\partial p_\mathbf{s}} \\
    &= - \log(p_{\mathbf{s}}) - 1 \notag
\end{align*}

\noindent
For the second term on the right-hand side we find

\begin{align*}
    \lambda_0 \left( \frac{\partial \left( \sum_{\mathbf{s}} p(\mathbf{s}) - 1 \right)}{\partial p_\mathbf{s}}\right) &= \lambda_0 \left( \frac{\partial \sum_{\mathbf{s}} p(\mathbf{s})}{\partial p_\mathbf{s}} -  \frac{\partial \: 1}{{\partial p_\mathbf{s}}}\right) \\
    &= \lambda_0 \left(\frac{\partial p_\mathbf{s}}{\partial p_\mathbf{s}} - 0 \right) \\
    &= \lambda_0 \notag
\end{align*}

\noindent
For the last term on the right-hand side we find

\begin{align*}
    \sum_{\mu \in \mathcal{M}} \alpha_\mu \left( \frac{\partial \left( \sum_{\mathbf{s}} \phi^\mu \: p(\mathbf{s}) - \expval{\phi^\mu}_{data} \right)}{\partial p_\mathbf{s}} \right) &= \sum_{\mu \in \mathcal{M}} \alpha_\mu \left( \frac{\partial \sum_{\mathbf{s}} \phi^\mu(\mathbf{s}) p(\mathbf{s})}{\partial p_\mathbf{s}} - \frac{\partial \expval{\phi^\mu}_{data}}{\partial p_\mathbf{s}} \right)\\
    &= \sum_{\mu \in \mathcal{M}} \alpha_\mu \left( \frac{\partial \phi^\mu(\mathbf{s}) p_{\mathbf{s}}}{\partial p_\mathbf{s}} - 0\right) \\
    &= \sum_{\mu \in \mathcal{M}} \alpha_\mu \phi^\mu(\mathbf{s})
\end{align*}

\noindent
Then, the final expression for the derivative of $\mathcal{L}[p(\mathbf{s})]$ with respect to $p(\mathbf{s})$ is

\begin{equation}
    \frac{\partial \mathcal{L}[\mathbf{p}]}{\partial p_\mathbf{s}} = - \log(p_{\mathbf{s}}) - 1 + \lambda_0 + \sum_{\mu \in \mathcal{M}} \alpha_\mu \phi^\mu(\mathbf{s}).
\end{equation}

\noindent
Setting the derivative of $\mathcal{L}[p(\mathbf{s})]$ with respect to $p(\mathbf{s})$ equal to zero gives the desired form of the probability distribution.

\begin{align*}
    0 &= \frac{\partial \mathcal{L}[\mathbf{p}]}{\partial p_\mathbf{s}}\\
    0 &= - \log(p_{\mathbf{s}}) - 1 + \lambda_0 + \sum_{\mu \in \mathcal{M}} \alpha_\mu \phi^\mu(\mathbf{s}) \\
    \log(p_{\mathbf{s}}) &= - 1 + \lambda_0 + \sum_{\mu \in \mathcal{M}} \alpha_\mu \phi^\mu(\mathbf{s})\\
    p_{\mathbf{s}} &= e^{\lambda_0 - 1 + \sum_{\mu \in \mathcal{M}} \alpha_\mu \phi^\mu(\mathbf{s})}\\
    &= e^{\lambda_0 - 1} e^{ \sum_{\mu \in \mathcal{M}} \alpha_\mu \phi^\mu(\mathbf{s})}
\end{align*}

\noindent
This expression has the same form as the expression in Equation \ref{eq:prob_distr}. The lagrange multipliers $\alpha_\mu$ have a one-to-one correspondence to interaction strength, $g_\mu$.
Then, the expression for the partition function becomes

\begin{equation*}
    Z = e^{1 - \lambda_0}.
\end{equation*}

\noindent
Knowing that the probability distribution is normalized this expression can be rewritten,

\begin{align*}
  &\sum_{\mathbf{s}} p_{\mathbf{s}} = 1, \\
  &\sum_{\mathbf{s}}  e^{\lambda_0 - 1} e^{\sum_{\mu \in \mathcal{M}} g_\mu \phi^\mu(\mathbf{s})} = 1, \\
  &e^{\lambda_0 - 1} = \frac{1}{\sum_{\mathbf{s}} e^{\sum_{\mu \in \mathcal{M}} g_\mu \phi^\mu(\mathbf{s})}},
\end{align*}

\noindent
which yields the final form of the probability distribution

\begin{equation}
  p_{\mathbf{g}}(\mathbf{s}) = \frac{e^{\sum_{\mu \in \mathcal{M}} g_\mu \phi^\mu(\mathbf{s})}}{\sum_{\mathbf{s}} e^{\sum_{\mu \in \mathcal{M}} g_\mu \phi^\mu(\mathbf{s})}}.
\end{equation}

\begin{landscape}
\newgeometry{top=2.5cm, left=2cm,bottom=1cm, right=2cm}
\subsection{Spin operator matrix three-state system with two spins} \label{sec:discrete_spin_op_matrix}

\begin{align*}
    & \hspace{-.2cm} \scriptstyle \mathbf{S}^{(3)} \otimes \mathbf{S}^{(3)} = \\
    & \hspace{-.3cm} \begin{bmatrix}
        \phi^0(\alpha=0) & \phi^1(\alpha=0) & {(\phi^1)}^2(\alpha=0)\\
        \phi^0(\alpha=1) & \phi^1(\alpha=1) & {(\phi^1)}^2(\alpha=1)\\
        \phi^0(\alpha=2) & \phi^1(\alpha=2) & {(\phi^1)}^2(\alpha=2)\\
    \end{bmatrix} \otimes \begin{bmatrix}
        \phi^0(\alpha=0) & \phi^1(\alpha=0) & {(\phi^1)}^2(\alpha=0)\\
        \phi^0(\alpha=1) & \phi^1(\alpha=1) & {(\phi^1)}^2(\alpha=1)\\
        \phi^0(\alpha=2) & \phi^1(\alpha=2) & {(\phi^1)}^2(\alpha=2)\\
    \end{bmatrix}=\\
    & \hspace{-.3cm} \begin{bmatrix}
        \scriptstyle \phi^0(\alpha_2=0) \phi^0(\alpha_1=0) & \hspace{-4pt} \scriptstyle \phi^0(\alpha_2=0) \phi^1(\alpha_1=0) & \hspace{-4pt} \scriptstyle \phi^0(\alpha_2=0) {(\phi^1)}^2(\alpha_1=0) & \hspace{-4pt} \scriptstyle \phi^1(\alpha_2=0) \phi^0(\alpha_1=0) & \hspace{-4pt} \scriptstyle \phi^1(\alpha_2=0) \phi^1(\alpha_1=0) & \hspace{-4pt} \scriptstyle \phi^1(\alpha_2=0) {(\phi^1)}^2(\alpha_1=0) & \hspace{-4pt} \scriptstyle {(\phi^1)}^2(\alpha_2=0) \phi^0(\alpha_1=0) & \hspace{-4pt} \scriptstyle {(\phi^1)}^2(\alpha_2=0) \phi^1(\alpha_1=0) & \hspace{-4pt} \scriptstyle {(\phi^1)}^2(\alpha_2=0) {(\phi^1)}^2(\alpha_1=0) \\ 
        \scriptstyle \phi^0(\alpha_2=0) \phi^0(\alpha_1=1) & \hspace{-4pt} \scriptstyle \phi^0(\alpha_2=0) \phi^1(\alpha_1=1) & \hspace{-4pt} \scriptstyle \phi^0(\alpha_2=0) {(\phi^1)}^2(\alpha_1=1) & \hspace{-4pt} \scriptstyle \phi^1(\alpha_2=0) \phi^0(\alpha_1=1) & \hspace{-4pt} \scriptstyle \phi^1(\alpha_2=0) \phi^1(\alpha_1=1) & \hspace{-4pt} \scriptstyle \phi^1(\alpha_2=0) {(\phi^1)}^2(\alpha_1=1) & \hspace{-4pt} \scriptstyle {(\phi^1)}^2(\alpha_2=0) \phi^0(\alpha_1=1) & \hspace{-4pt} \scriptstyle {(\phi^1)}^2(\alpha_2=0) \phi^1(\alpha_1=1) & \hspace{-4pt} \scriptstyle {(\phi^1)}^2(\alpha_2=0) {(\phi^1)}^2(\alpha_1=1) \\
        \scriptstyle \phi^0(\alpha_2=0) \phi^0(\alpha_1=2) & \hspace{-4pt} \scriptstyle \phi^0(\alpha_2=0) \phi^1(\alpha_1=2) & \hspace{-4pt} \scriptstyle \phi^0(\alpha_2=0) {(\phi^1)}^2(\alpha_1=2) & \hspace{-4pt} \scriptstyle \phi^1(\alpha_2=0) \phi^0(\alpha_1=2) & \hspace{-4pt} \scriptstyle \phi^1(\alpha_2=0) \phi^1(\alpha_1=2) & \hspace{-4pt} \scriptstyle \phi^1(\alpha_2=0) {(\phi^1)}^2(\alpha_1=2) & \hspace{-4pt} \scriptstyle {(\phi^1)}^2(\alpha_2=0) \phi^0(\alpha_1=2) & \hspace{-4pt} \scriptstyle {(\phi^1)}^2(\alpha_2=0) \phi^1(\alpha_1=2) & \hspace{-4pt} \scriptstyle {(\phi^1)}^2(\alpha_2=0) {(\phi^1)}^2(\alpha_1=2) \\ 
        \scriptstyle \phi^0(\alpha_2=1) \phi^0(\alpha_1=0) & \hspace{-4pt} \scriptstyle \phi^0(\alpha_2=1) \phi^1(\alpha_1=0) & \hspace{-4pt} \scriptstyle \phi^0(\alpha_2=1) {(\phi^1)}^2(\alpha_1=0) & \hspace{-4pt} \scriptstyle \phi^1(\alpha_2=1) \phi^0(\alpha_1=0) & \hspace{-4pt} \scriptstyle \phi^1(\alpha_2=1) \phi^1(\alpha_1=0) & \hspace{-4pt} \scriptstyle \phi^1(\alpha_2=1) {(\phi^1)}^2(\alpha_1=0) & \hspace{-4pt} \scriptstyle {(\phi^1)}^2(\alpha_2=1) \phi^0(\alpha_1=0) & \hspace{-4pt} \scriptstyle {(\phi^1)}^2(\alpha_2=1) \phi^1(\alpha_1=0) & \hspace{-4pt} \scriptstyle {(\phi^1)}^2(\alpha_2=1) {(\phi^1)}^2(\alpha_1=0) \\ 
        \scriptstyle \phi^0(\alpha_2=1) \phi^0(\alpha_1=1) & \hspace{-4pt} \scriptstyle \phi^0(\alpha_2=1) \phi^1(\alpha_1=1) & \hspace{-4pt} \scriptstyle \phi^0(\alpha_2=1) {(\phi^1)}^2(\alpha_1=1) & \hspace{-4pt} \scriptstyle \phi^1(\alpha_2=1) \phi^0(\alpha_1=1) & \hspace{-4pt} \scriptstyle \phi^1(\alpha_2=1) \phi^1(\alpha_1=1) & \hspace{-4pt} \scriptstyle \phi^1(\alpha_2=1) {(\phi^1)}^2(\alpha_1=1) & \hspace{-4pt} \scriptstyle {(\phi^1)}^2(\alpha_2=1) \phi^0(\alpha_1=1) & \hspace{-4pt} \scriptstyle {(\phi^1)}^2(\alpha_2=1) \phi^1(\alpha_1=1) & \hspace{-4pt} \scriptstyle {(\phi^1)}^2(\alpha_2=1) {(\phi^1)}^2(\alpha_1=1) \\ 
        \scriptstyle \phi^0(\alpha_2=1) \phi^0(\alpha_1=2) & \hspace{-4pt} \scriptstyle \phi^0(\alpha_2=1) \phi^1(\alpha_1=2) & \hspace{-4pt} \scriptstyle \phi^0(\alpha_2=1) {(\phi^1)}^2(\alpha_1=2) & \hspace{-4pt} \scriptstyle \phi^1(\alpha_2=1) \phi^0(\alpha_1=2) & \hspace{-4pt} \scriptstyle \phi^1(\alpha_2=1) \phi^1(\alpha_1=2) & \hspace{-4pt} \scriptstyle \phi^1(\alpha_2=1) {(\phi^1)}^2(\alpha_1=2) & \hspace{-4pt} \scriptstyle {(\phi^1)}^2(\alpha_2=1) \phi^0(\alpha_1=2) & \hspace{-4pt} \scriptstyle {(\phi^1)}^2(\alpha_2=1) \phi^1(\alpha_1=2) & \hspace{-4pt} \scriptstyle {(\phi^1)}^2(\alpha_2=1) {(\phi^1)}^2(\alpha_1=2) \\
        \scriptstyle \phi^0(\alpha_2=2) \phi^0(\alpha_1=0) & \hspace{-4pt} \scriptstyle \phi^0(\alpha_2=2) \phi^1(\alpha_1=0) & \hspace{-4pt} \scriptstyle \phi^0(\alpha_2=2) {(\phi^1)}^2(\alpha_1=0) & \hspace{-4pt} \scriptstyle \phi^1(\alpha_2=2) \phi^0(\alpha_1=0) & \hspace{-4pt} \scriptstyle \phi^1(\alpha_2=2) \phi^1(\alpha_1=0) & \hspace{-4pt} \scriptstyle \phi^1(\alpha_2=2) {(\phi^1)}^2(\alpha_1=0) & \hspace{-4pt} \scriptstyle {(\phi^1)}^2(\alpha_2=2) \phi^0(\alpha_1=0) & \hspace{-4pt} \scriptstyle {(\phi^1)}^2(\alpha_2=2) \phi^1(\alpha_1=0) & \hspace{-4pt} \scriptstyle {(\phi^1)}^2(\alpha_2=2) {(\phi^1)}^2(\alpha_1=0) \\
        \scriptstyle \phi^0(\alpha_2=2) \phi^0(\alpha_1=1) & \hspace{-4pt} \scriptstyle \phi^0(\alpha_2=2) \phi^1(\alpha_1=1) & \hspace{-4pt} \scriptstyle \phi^0(\alpha_2=2) {(\phi^1)}^2(\alpha_1=1) & \hspace{-4pt} \scriptstyle \phi^1(\alpha_2=2) \phi^0(\alpha_1=1) & \hspace{-4pt} \scriptstyle \phi^1(\alpha_2=2) \phi^1(\alpha_1=1) & \hspace{-4pt} \scriptstyle \phi^1(\alpha_2=2) {(\phi^1)}^2(\alpha_1=1) & \hspace{-4pt} \scriptstyle {(\phi^1)}^2(\alpha_2=2) \phi^0(\alpha_1=1) & \hspace{-4pt} \scriptstyle {(\phi^1)}^2(\alpha_2=2) \phi^1(\alpha_1=1) & \hspace{-4pt} \scriptstyle {(\phi^1)}^2(\alpha_2=2) {(\phi^1)}^2(\alpha_1=1) \\
        \scriptstyle \phi^0(\alpha_2=2) \phi^0(\alpha_1=2) & \hspace{-4pt} \scriptstyle \phi^0(\alpha_2=2) \phi^1(\alpha_1=2) & \hspace{-4pt} \scriptstyle \phi^0(\alpha_2=2) {(\phi^1)}^2(\alpha_1=2) & \hspace{-4pt} \scriptstyle \phi^1(\alpha_2=2) \phi^0(\alpha_1=2) & \hspace{-4pt} \scriptstyle \phi^1(\alpha_2=2) \phi^1(\alpha_1=2) & \hspace{-4pt} \scriptstyle \phi^1(\alpha_2=2) {(\phi^1)}^2(\alpha_1=2) & \hspace{-4pt} \scriptstyle {(\phi^1)}^2(\alpha_2=2) \phi^0(\alpha_1=2) & \hspace{-4pt} \scriptstyle {(\phi^1)}^2(\alpha_2=2) \phi^1(\alpha_1=2) & \hspace{-4pt} \scriptstyle {(\phi^1)}^2(\alpha_2=2) {(\phi^1)}^2(\alpha_1=2)  
    \end{bmatrix}\scriptstyle = \hspace{-.7cm}\\
    & \hspace{-.3cm} \begin{bmatrix}
        1 & 1 & 1 & 1 & 1 & 1 & 1 & 1 & 1 \\
        1 & e^{-\frac{2\pi i}{3}} & e^{-\frac{4\pi i}{3}} & 1 & e^{-\frac{2\pi i}{3}} & e^{-\frac{4\pi i}{3}} & 1 & e^{-\frac{2\pi i}{3}} & e^{-\frac{4\pi i}{3}} \\
        1 & e^{-\frac{4\pi i}{3}} & e^{-\frac{2\pi i}{3}} & 1 & e^{-\frac{4\pi i}{3}} & e^{-\frac{2\pi i}{3}} & 1 & e^{-\frac{4\pi i}{3}} & e^{-\frac{2\pi i}{3}} \\
        1 & 1 & 1 & e^{-\frac{2\pi i}{3}} & e^{-\frac{2\pi i}{3}} & e^{-\frac{2\pi i}{3}} & e^{-\frac{4\pi i}{3}} & e^{-\frac{4\pi i}{3}} & e^{-\frac{4\pi i}{3}} \\
        1 & e^{-\frac{2\pi i}{3}} & e^{-\frac{4\pi i}{3}} & e^{-\frac{2\pi i}{3}} & e^{-\frac{4\pi i}{3}} & 1 & e^{-\frac{4\pi i}{3}} & 1 & e^{-\frac{2\pi i}{3}} \\
        1 & e^{-\frac{4\pi i}{3}} & e^{-\frac{2\pi i}{3}} & e^{-\frac{2\pi i}{3}} & 1 & e^{-\frac{4\pi i}{3}} & e^{-\frac{4\pi i}{3}} & e^{-\frac{2\pi i}{3}} & 1 \\
        1 & 1 & 1 & e^{-\frac{4\pi i}{3}} & e^{-\frac{4\pi i}{3}} & e^{-\frac{4\pi i}{3}} & e^{-\frac{2\pi i}{3}} & e^{-\frac{2\pi i}{3}} & e^{-\frac{2\pi i}{3}} \\
        1 & e^{-\frac{2\pi i}{3}} & e^{-\frac{4\pi i}{3}} & e^{-\frac{4\pi i}{3}} & 1 & e^{-\frac{2\pi i}{3}} & e^{-\frac{2\pi i}{3}} & e^{-\frac{4\pi i}{3}} & 1 \\
        1 & e^{-\frac{4\pi i}{3}} & e^{-\frac{2\pi i}{3}} & e^{-\frac{4\pi i}{3}} & e^{-\frac{2\pi i}{3}} & 1 & e^{-\frac{2\pi i}{3}} & 1 & e^{-\frac{4\pi i}{3}} \\
    \end{bmatrix}\\
\end{align*}
\restoregeometry
\end{landscape}

\subsection{Expected values at maximum log-likelihood} \label{sec:max_log_likelihood}

Setting the derivative of the log-likelihood in Equation \ref{eq:log_likelihood} with respect to the model parameters $g_\mu$ equal to zero gives,

\begin{align*}
    0 &= \frac{\partial \log P(\mathbf{\hat{s}} | \mathbf{g}, \mathcal{M})}{\partial g_\mu}, \\
    &= N \frac{\partial  \mathbf{g} \cdot \mathbf{\phi}(\mathbf{\hat{s}}) - \log {Z_\mathbf{g}(\mathcal{M})}}{\partial g_\mu}, \\
    &= N \left[ \phi^\mu(\mathbf{\hat{s}}) - \frac{1}{Z_\mathbf{g}(\mathcal{M})} \frac{\partial {Z_\mathbf{g}(\mathcal{M})}}{\partial g_\mu} \right], \\
    &= N \left[ \phi^\mu(\mathbf{\hat{s}}) - \frac{1}{Z_\mathbf{g}(\mathcal{M})} \frac{\partial \sum_{\mathbf{s} \in \mathbf{\hat{s}}} e^{\left(\sum_{\mu \in \mathcal{M}} g_\mu \phi^\mu(\mathbf{s}) \right)}}{\partial g_\mu} \right], \\
    &= N \left[ \phi^\mu(\mathbf{\hat{s}}) - \frac{1}{Z_\mathbf{g}(\mathcal{M})} \sum_{\mathbf{s} \in \mathbf{\hat{s}}} \phi^\mu(\mathbf{s}) e^{\left(\sum_{\mu \in \mathcal{M}} g_\mu \phi^\mu(\mathbf{s}) \right)} \right], \\
    &= N \left[ \phi^\mu(\mathbf{\hat{s}}) - \sum_{\mathbf{s} \in \mathbf{\hat{s}}} \phi^\mu(\mathbf{s}) P(\mathbf{s} | \mathbf{g}, \mathcal{M})\right],
\end{align*}

\noindent
which shows that at the maximum of the log-likelihood

\begin{equation}
  \phi^\mu(\mathbf{\hat{s}}) = \sum_{\mathbf{s} \in \mathbf{\hat{s}}} \phi^\mu(\mathbf{s}) P(\mathbf{s} | \mathbf{g}, \mathcal{M}).
\end{equation}

\subsection{Expression for evidence using Laplace's method} \label{sec:laplace}

Because the value of the integral in Equation \ref{eq:evidence} is dominated by a single contribution at the maximum, we can replace the term in the exponential by its second-order Taylor expansion around the model parameters, $\mathbf{g}^\star$, that maximize the evidence.

\begin{align*}
  N  \left[ \mathbf{g} \cdot \mathbf{\phi}(\mathbf{\hat{s}}) - \log {Z_\mathbf{g}(\mathcal{M})} \right] \approx& \: N  \left[ \mathbf{g}^\star \cdot \mathbf{\phi}(\mathbf{\hat{s}}) - \log {Z_\mathbf{g^\star}(\mathcal{M})} \right] + (\mathbf{g} - \mathbf{g}^\star)^\intercal \nabla \bigl[ N  \left[ \mathbf{g} \cdot \mathbf{\phi}(\mathbf{\hat{s}}) - \log {Z_\mathbf{g}(\mathcal{M})} \right] \bigr](\mathbf{g}^\star) \\
&+ \frac{1}{2}  (\mathbf{g} - \mathbf{g}^\star)^\intercal \nabla^2 \bigl[ N  \left[ \mathbf{g} \cdot \mathbf{\phi}(\mathbf{\hat{s}}) - \log {Z_\mathbf{g}(\mathcal{M})} \right] \bigr] (\mathbf{g}^\star) (\mathbf{g} - \mathbf{g}^\star) \\
=& \log P(\mathbf{\hat{s}} | \mathbf{g}^\star, \mathcal{M}) - \frac{1}{2}  (\mathbf{g} - \mathbf{g}^\star)^\intercal \nabla^2 \bigl[ N \log {Z_\mathbf{g}(\mathcal{M})} \bigr] (\mathbf{g}^\star) (\mathbf{g} - \mathbf{g}^\star)
%=& \log P(\mathbf{\hat{s}} | \mathbf{g}^\star, \mathcal{M}) - \frac{N}{2} (\mathbf{g} - \mathbf{g}^\star)^\intercal \mathbb{J}(\mathbf{g}^\star)(\mathbf{g} - \mathbf{g}^\star)
\end{align*}

%\begin{equation*}
%  P_0(\mathbf{g}|\mathcal{M}) \approx \: P_0(\mathbf{g}^\star|\mathcal{M}) +  (\mathbf{g} - \mathbf{g}^\star)^\intercal \nabla \bigl[ P_0(\mathbf{g}|\mathcal{M})\bigr](\mathbf{g}^\star) + \frac{1}{2} (\mathbf{g} - \mathbf{g}^\star)^\intercal \nabla^2 \bigl[ P_0(\mathbf{g}|\mathcal{M}) \bigr] (\mathbf{g}^\star) (\mathbf{g} - \mathbf{g}^\star)
%\end{equation*}

\noindent
Plugging this expansion into the expression for the evidence and using the definition of the Fisher information matrix, which has as entries

\begin{equation}
    \mathbb{J}_{\mu\nu}(\mathbf{g}) = \frac{\partial^2 \log {Z_\mathbf{g}(\mathcal{M})}}{\partial g_\mu \partial g_\nu},
\end{equation}

\noindent
gives the following expression for the evidence.

\begin{align*}
    P(\mathbf{\hat{s}}|\mathcal{M}) \approx& \int d\mathbf{g} \: e^{\log P(\mathbf{\hat{s}} | \mathbf{g}^\star, \mathcal{M}) - \frac{N}{2}  (\mathbf{g} - \mathbf{g}^\star)^\intercal \mathbb{J} (\mathbf{g}^\star) (\mathbf{g} - \mathbf{g}^\star)} P_0(\mathbf{g}|\mathcal{M}) \\
   =& \int d\mathbf{g} \: P(\mathbf{\hat{s}} | \mathbf{g}^\star, \mathcal{M})  e^{-\frac{N}{2}  (\mathbf{g} - \mathbf{g}^\star)^\intercal \mathbb{J} (\mathbf{g}^\star) (\mathbf{g} - \mathbf{g}^\star)} P_0(\mathbf{g}|\mathcal{M}) \\
   =& P(\mathbf{\hat{s}} | \mathbf{g}^\star, \mathcal{M}) P_0(\mathbf{g}^\star|\mathcal{M}) \int d\mathbf{g} \: e^{-\frac{N}{2}  (\mathbf{g} - \mathbf{g}^\star)^\intercal \mathbb{J} (\mathbf{g}^\star) (\mathbf{g} - \mathbf{g}^\star)} \\
\end{align*}

%\begin{align*}
%    P(\mathbf{\hat{s}}|\mathcal{M}) \approx& \int d\mathbf{g} \: e^{\log P(\mathbf{\hat{s}} | \mathbf{g}^\star, \mathcal{M}) - \frac{N}{2}  (\mathbf{g} - \mathbf{g}^\star)^\intercal \mathbb{J} (\mathbf{g}^\star) (\mathbf{g} - \mathbf{g}^\star)} \\
%    & \left( P_0(\mathbf{g}^\star|\mathcal{M}) +  (\mathbf{g} - \mathbf{g}^\star)^\intercal \nabla \bigl[ P_0(\mathbf{g}|\mathcal{M})\bigr](\mathbf{g}^\star) + \frac{1}{2} (\mathbf{g} - \mathbf{g}^\star)^\intercal \nabla^2 \bigl[ P_0(\mathbf{g}|\mathcal{M}) \bigr] (\mathbf{g}^\star) (\mathbf{g} - \mathbf{g}^\star) \right) \\
%   =& \int d\mathbf{g} \: P(\mathbf{\hat{s}} | \mathbf{g}^\star, \mathcal{M})  e^{-\frac{N}{2}  (\mathbf{g} - \mathbf{g}^\star)^\intercal \mathbb{J} (\mathbf{g}^\star) (\mathbf{g} - \mathbf{g}^\star)} \\
%   &\left( P_0(\mathbf{g}^\star|\mathcal{M}) +  (\mathbf{g} - \mathbf{g}^\star)^\intercal \nabla \bigl[ P_0(\mathbf{g}|\mathcal{M})\bigr](\mathbf{g}^\star) + \frac{1}{2} (\mathbf{g} - \mathbf{g}^\star)^\intercal \nabla^2 \bigl[ P_0(\mathbf{g}|\mathcal{M}) \bigr] (\mathbf{g}^\star) (\mathbf{g} - \mathbf{g}^\star) \right) \\
%   =& P(\mathbf{\hat{s}} | \mathbf{g}^\star, \mathcal{M}) P_0(\mathbf{g}^\star|\mathcal{M}) \int d\mathbf{g} \: e^{-\frac{N}{2}  (\mathbf{g} - \mathbf{g}^\star)^\intercal \mathbb{J} (\mathbf{g}^\star) (\mathbf{g} - \mathbf{g}^\star)} \\
%   &\left(1 + \frac{1}{P_0(\mathbf{g}^\star|\mathcal{M})} (\mathbf{g} - \mathbf{g}^\star)^\intercal \nabla \bigl[ P_0(\mathbf{g}|\mathcal{M})\bigr](\mathbf{g}^\star) + \frac{1}{2 P_0(\mathbf{g}^\star|\mathcal{M})} (\mathbf{g} - \mathbf{g}^\star)^\intercal \nabla^2 \bigl[ P_0(\mathbf{g}|\mathcal{M}) \bigr] (\mathbf{g}^\star) (\mathbf{g} - \mathbf{g}^\star) \right)
%\end{align*}

\noindent
The integral is a multidimensional Gaussian integral. Because the Fisher information matrix is symmetric, we can decompose it as,

\begin{equation*}
    \mathbb{J} = Q \Lambda Q^\intercal,
\end{equation*}

\noindent
where Q is an orthonormal matrix that has the linearly independent eigenvectors of $\mathbb{J}$ as columns. Defining a new variable $\mathbf{x} \equiv Q^\intercal (\mathbf{g} - \mathbf{g}^\star)$, allows us to write the integral as a product of 1D Gaussian integrals.

\begin{align*}
    \int d\mathbf{g} \: e^{-\frac{N}{2}  (\mathbf{g} - \mathbf{g}^\star)^\intercal \mathbb{J} (\mathbf{g}^\star) (\mathbf{g} - \mathbf{g}^\star)} &= \int d\mathbf{g} \: e^{-\frac{N}{2}  (\mathbf{g} - \mathbf{g}^\star)^\intercal Q \Lambda Q^\intercal (\mathbf{g} - \mathbf{g}^\star)}\\
    &= \int d\mathbf{x} \: e^{-\frac{N}{2}  \mathbf{x}^\intercal \Lambda \mathbf{x}} \\
    &= \prod_{\mu = 1}^{|\mathcal{M}|} \int dx_\mu e^{-\frac{N}{2} \lambda_\mu x_\mu^2} \\
    &= \prod_{\mu = 1}^{|\mathcal{M}|} \sqrt{\frac{2 \pi}{N \lambda_\mu}} \\
    &= \left(\frac{2 \pi}{N}\right)^\frac{|\mathcal{M}|}{2} \frac{1}{\sqrt{\text{det } \mathbb{J}(\mathbf{g}^\star)}}
\end{align*}

\noindent
Substituting this into the expression for the log-evidence gives

\begin{align*}
    \log P(\mathbf{\hat{s}}|\mathcal{M}) &\approx \log \left[ P(\mathbf{\hat{s}} | \mathbf{g}^\star, \mathcal{M}) P_0(\mathbf{g}^\star|\mathcal{M}) \left(\frac{2 \pi}{N}\right)^\frac{|\mathcal{M}|}{2}  \frac{1}{\sqrt{\text{det } \mathbb{J}(\mathbf{g}^\star)}} \right]\\
    &= \log P(\mathbf{\hat{s}} | \mathbf{g}^\star, \mathcal{M}) -  \frac{|\mathcal{M}|}{2} \log \frac{N}{2 \pi} - \log \left[ \frac{\sqrt{\text{det } \mathbb{J}(\mathbf{g}^\star)}}{P_0(\mathbf{g}^\star|\mathcal{M})} \right]
\end{align*}


%\printbibliography
%-------------------------------------------------------------------------------
%	REFERENCES
%-------------------------------------------------------------------------------

%-------------------------------------------------------------------------------
%	BIJLAGEN 
%-------------------------------------------------------------------------------



%-------------------------------------------------------------------------------
\end{document}


